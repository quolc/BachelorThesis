% ビットレートの要件 ... double*3を入れるには200bpsは必須
Considering practical usage of annotations for content editing, we thought that in minimum 200 bps is required as the payload rate of watermarking, because if an user wants to record a value of a sensor (e.g. GPS) consists of two or three double-precision floating-point number every seconds, 8 (bit/byte) * 8 (bytes) * 3 = 192 (bits) should be able to embedded in one second.
% ヘッダや透かし間ギャップを考慮すると400bpsは欲しい
Taking overhead from packet header and required gap between watermarks for stability of decoding into account, at least 400 bps is desirable as the gross bit rate of our watermarking technique.

% 18kHz以上の帯域幅を有効に使うため、OFDMを採用した
To satisfy this requirement for bit rate, we adopt orthogonal frequency-division multiplexing (OFDM) as the modulation method to make full use of available bandwidth over 18 kHz, following the idea of Matsuoka's {\it Acoustic OFDM} \cite{matsuoka2008acoustic}.
% OFDMとは
OFDM is a specialized variety of frequency-division multiplexing (FDM) that uses a number of sub-carriers with different frequencies to transmit information in parallel.
In OFDM, carrier frequencies are selected in orthogonal, in other words, signals of sub-carriers are independent of and do not interfere each other.
This characteristics realize closer spaces between carrier frequencies and transmitting larger number of sub-carriers than ordinary FDM in certain bandwidth, therefore using OFDM increases bit rate of transmission.

% 17.9kHz~20khzの8並列で送ってる
AnnoTone uses 8 sub-carriers with equally spaced frequencies between 17900 Hz and 20000 Hz (17900 Hz, 18200 Hz, 18500 Hz, 18800 Hz, 19100 Hz, 19400 Hz, 19700 Hz, 20000 Hz) thus one byte can be sent in one parallel signal.
% サブキャリアのモジュレーションについて
Each sub-carrier signal is modulated using differential binary phase shift keying (D-BPSK) in 100 bps (10 milliseconds per unit signal).
D-BPSK represents binary data in a series of unit signals by phase shifting of certain frequency component.
A unit signal represents `` 1 '' when the phase of the signal is same as that of previous one, it represents `` 0 '' when the phase of the signal is shifted $\pi$ radians from that of previous one.
Using OFDM reduces computational cost of decoding significantly, enabling D-BPSK demodulation of each sub-carrier to be done in parallel without applying any filter such as band pass filter (BPF).
% 合計のビットレートは
In total, gross bit rate is 800 bps in our system, which satisfies the requirement mentioned above.
