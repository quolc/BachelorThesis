\begin{jabstract}
音声透かしとは、人間には検知しにくい形態で音声信号中に付加情報を埋め込む技術である。
従来の音声透かし技術は既存コンテンツの知的財産権保護および音声放送への文字情報重畳を主たる応用としており、透かしの埋め込みはポストプロダクションの最終段階で行われる。
一方、音声透かしは耐編集性やコンテンツとの同期性に優れていることから、映像制作時に用いるアノテーションなどのコンテナとしても適切であると考えられる。
本論文では、デジタルデータを音声透かしに変換してカメラの近くに置いたデバイスのスピーカーから発音させることで、編集時に有用な情報をビデオカメラによる映像収録時に音声信号中に埋め込む手法を提案する。
本手法で埋め込まれる音声透かしは完全に不可聴ではないが、デジタルフィルタを用いることで音質を大きく落とすことなく音声信号から取り除けることをユーザーテストにより確認した。
また、音声透かし埋め込みの信頼性や音声データの変換に対する透かしの耐久性を、実際の利用場面を想定した実験により評価し、本手法が高い実用性を備えることを検証した。
論文中では提案手法を使用した複数の映像編集アプリケーションを例示し、アノテーションを用いることで実現される様々な新しい映像制作技法の可能性を議論する。
\end{jabstract}
