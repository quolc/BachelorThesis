% 共通で使用できるAndroid用ライブラリを作成した
% ライブラリは次のようなアルゴリズムを使用している
In the system of AnnoTone, watermark signals of annotations are generated and transmitted from a smartphone.
We implemented a software library that provides essential functions for watermarking for creating android applications for AnnoTone system.
Our library uses following algorithm to generate watermark signals from an annotation data.

% 入力データに対してどのように処理を行うか
% annotation -> byte array -> packet -> wave data
% 位相変化時の低周波成分ノイズ発生を抑えるために窓関数を掛ける
First, an annotation of any kind of data (e.g. GPS position, a situation of ball game) is serialized to a payload byte array following the arrangement manner mentioned in the previous chapter.
Secondly, a data stream of a packet is generated from the payload by calculating CRC-16 checksum.
Thirdly, pcm wave for each sub-carrier are calculated using next formulae.

$\theta(n, m)$ is the phase of the wave of the m-th sub-carrier at the n-th data frame, and $d(n, m)$ is the value of m-th bit from LSB of n-th byte of the packet.

\begin{align}
\theta(0, m) &= 0 \\
\theta(n, m) &= \begin{cases}
	\theta(n-1, m) + \pi & (d(n, m) = 0) \\
	\theta(n-1, m) & (d(n, m) = 1)
\end{cases}
\end{align}

$s(t, m)$ is the sample value of the wave of the m-th sub-carrier at timestamp $t$ (s) counted from the beginning of the wave. $f(m)$ is the frequency (Hz) of the m-th sub-carrier.

\begin{align}
s(t, m) &= w( \sin{(2 \pi t \cdot f(m) + \theta(\lfloor \frac{t}{0.01} \rfloor, m))}, t)
\end{align}
$w(x, t)$ is an envelope filtering function to reduce the occurrence of low frequency noise at borders where the phase is shifted defined as follows.

\begin{align}
w(x, t) &= \sqrt{(0.5 - 0.5\cos{( 2 \pi \frac{t \bmod 0.01}{0.01} )})}
\end{align}

Finally, watermark signal $x(t)$ to transmit is generated by mixing all sub-carrier signals as $ x(t) = \sum^{8}_{m=1} s(t, m)$.
