\begin{eabstract}
Digital audio watermarking is a technique to embed additional data in audio signal in imperceptible form to human.
An audio watermark has been mainly used for identifying intellectual property rights of existing digital contents and superimposing text information like URL on audio broadcasting, thus it is embedded at the final stage of post-production.
On the other hand, an audio watermark could be suitable to contain annotations in editing process because of its robustness for editing and synchrony to the content.
We propose an audio watermarking technique that enables embedding useful information for audio/video editing in an audio sequence while shooting with a common video camera by synthesizing audio watermark signals from digital data and transmitting them from a speaker of a device near the camera.
In this paper, we also present several examples of video editing application using this technique, and discuss its superiority to other techniques with similar purposes.
\end{eabstract}
