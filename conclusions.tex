\chapter{Conclusions and Future Work}

% [contribution]
% 本論文で我々はAnnoToneというビデオアノテーションシステムとその具体的な応用例を示した。
In this paper, we proposed a video annotating system named AnnoTone, and practical video editing applications using this system.
% 我々のコントリビューションは以下の3点に要約できる
Our contributions in this research would be summarized in following three points.
\begin{itemize}
% 1. audio watermarkingを用いることで、機材制約の少ない動画への情報注釈手法を提案した
\item We presented a novel method to annotate a video in recording-time with small hardware constraints using digital audio watermarking technique.
% 2. アノテーションのために特化したaudio watermarking schemeを開発し、性能を評価した
\item We developed a DTMF-based audio watermarking scheme suited to annotate videos for supporting movie content creation, and conducted quantitative evaluations on its performance in actual situations.
% 3. 録画時アノテーションを用いた新しい動画制作手法を提案し、実用性のあるアプリケーションを複数示した
\item We showed practicability of our annotation system by suggesting video editing applications that utilize contextual information embedded in videos to provide rich user interfaces and automate editing processes.
\end{itemize}
% 動画編集において意味論を反映した様々な新しいインタラクション手法を実現できる
Our system can be used to provide various interaction styles for video editing that reflect semantics of videos and be more intuitive than traditional timeline-based editing systems. 
% 我々のシステムは少ない機材制約のために誰でもすぐに利用することができる
Viewing from the standpoint of accessibility and versatility, our system has a large advantage owing to its small hardware constraint, and can be used by almost every creators with different purposes to support their work.

% memo (limitations)
However, it seems true that there are some limitations for using our method in a professional work.
% 1. プロフェッショナルレベルのプロダクションを作るには、watermarkの痕跡を完全に消すことは出来ない問題がある
Firstly, still in most cases they are imperceptible to human, embedded watermarks are not completely transparent, and sometimes their high frequency sounds give strange feelings to a listener.
Although our watermark deletion program can reduce the watermark sounds to unnoticeable volume, part of them can be detected after being deletion using a common digital audio editor like Audacity \cite{audacity}.
This seems somewhat unsuitable for distributed productions.
On the other hand, using other methods like downsampling can completely remove watermarks, however they inevitably cause unignorable changes on listening quality.
To solve these problems, we should develop more elaborated technique to completely eliminate watermarks in future, for example, utilizing existing noise canceling technologies.

% 2. データを音声として出力するには遅延があるので、厳密なタイミング同期を要求する用途には使いにくい
% ただし、同じハードウェアセッティング、ペイロード長に対しての遅延はほぼ一定なので、事前に計測して引いてあげれば対応は可能
Delay from the calculation time for watermark signal generation and sound transmission restricts the precision of synchronization between a video sequence and embedded watermarks.
Because of the limitation, it is not suited for our technique to record the accurate time of instant events such as battings of baseball.
However, this delay could be partly compensated by estimating the duration of delay, since this is almost constant in fixed hardware setup and payload length.

% 3. 機材制約は全くないわけではない。スマートフォンのスピーカーは高音をノイズ無く出力出来なければならないし、
% カメラは高音をセンシングしてかつ適切なフォーマットで保存できる必要がある。
Our technique has a smaller hardware constraint compared with existing techniques with similar purposes as mentioned above, however, it is not zero.
The speaker of a smartphone should be able to transmit high frequency sounds above hearing range with minimum audible noise, and the microphone of a camera should record them without attenuation of signals and save the audio sequence in adequate quality.
We learned that some cheap instruments do not satisfy these requirements, for example, an out-of-date smartphone causes low frequency noises when it is transmitting a watermark, and a compact video camera records sound with a sampling rate below 44100 Hz.
Nevertheless, we think that this is not a major problem, because more and more high-performance instruments are provided in low price due to the fast advance of electronics.
