\begin{jabstract}
音声透かしとは、人間には検知しにくい形態で音声信号中に付加情報を埋め込む技術である。
従来の音声透かし技術は既存コンテンツの知的財産権保護および音声放送への文字情報重畳を主たる応用としており、透かしの埋め込みはポストプロダクションの最終段階で行われる。
一方、音声透かしは耐編集性やコンテンツとの同期性に優れていることから、映像制作時に用いるアノテーションなどのコンテナとしても適切であると考えられる。
本論文では、デジタルデータを音声透かしに変換してカメラの近くに置いたデバイスのスピーカーから発音させることで、編集時に有用な情報を通常のビデオカメラによる映像収録時に音声信号中に埋め込む手法を提案する。
また、本手法を使用した複数の映像編集アプリケーションを例示し、類似の目的を持つ他の技術に対する本手法の優越性を議論する。
\end{jabstract}