\begin{eabstract}
Digital audio watermarking is a technique to embed additional data in audio signal in imperceptible form to human.
An audio watermark has been mainly used for identifying intellectual property rights of existing digital contents and superimposing text information like URL on audio broadcasting, thus it has been embedded at the final stage of post-production.
On the other hand, audio watermarks may be suitable to contain annotations for movie making because of its robustness for editing and synchrony to the content.
We propose an audio watermarking technique that enables embedding useful information for audio/video editing in an audio sequence while shooting with a common video camera by synthesizing audio watermark signals from digital data and transmitting them from a speaker of a device near the camera.
Audio watermarks embedded by our technique are not completely inaudible, but verified by a user testing that they can be erased from an audio data with small sound quality loss by applying a digital filter.
We also evaluated the reliability of watermark embedding and the durability of watermarks against audio format conversions to show the high practicability of the proposed technique.
In this thesis, we present several examples of video-editing application using this technique, and discuss the possibilities of novel movie making systems realized by use of annotations.
\end{eabstract}
