% watermarkデコードのアルゴリズム
Watermarks of AnnoTone embedded in media files can be extracted using common demodulation methods for D-BPSK. We implemented a decoder library using following algorithm, which could be used by any kind of applications utilizing annotations for content editing.

% まず形式を変換する
First, any kind of input files such as MPEG-2 format movie are converted to pcm (.wav) file with 16 bit / 44100 Hz sampling format using an external audio/video converter such as ffmpeg. % reference付ける
% 波形を3.333msごとに区切る
Secondly, input signal is divided into short sections with the length of one third of that of a unit signal, this is about 3.33 milliseconds, so that any data frame embedded in the input signal contains at least two short sections.

% 区間ごとに各周波数・各位相の波を掛け合わせる
% 各周波数において相関が最大となる位相を計算する
After that, phase detection for each sub-carrier frequency is applied to section by section.
The phase of frequency $f$ at a section from $t_a$ to $t_b$ is approximated by calculating $n$ which maximizes following function $c(t_a, t_b, n)$.
$ts(t_a, t_b)$ is the set of timestamps of samples contained in the section from $t_a$ to $t_b$ and $x(t)$ is the sample value at the timestamp $t$.
\begin{align}
c(t_a, t_b, n) = \sum_{t \in ts(t_a, t_b)}
	\sin{( 2 \pi (t \cdot f + \frac{n}{32}))} \cdot x(t) \;\;\;\; (n = 1,2,...,32)
\end{align}
This function calculates correlation between the input signal and the sine wave with the frequency $f$ and a phase at $\frac{1}{16}$ radians intervals. It is maximized when the phase of the sub-carrier signal and that of sin wave correspond.

% 全ての周波数成分において、2区間で連続して誤差範囲で同じ位相が出現したらスタートビットと判定
% 以降3区間おきに各周波数成分の位相をチェックし、変化からD-BPSKデコードする
When phase shift between two successive sections is smaller than a certain amount, we used $\frac{\pi}{4}$, for all sub-carriers, that position is regarded as a start frame of a watermark.
After a start frame is detected, values for each data frames in the packet is calculated by seeing phase shifts for each sub-carriers between sections at intervals of three sections.
If a phase shift is smaller than $\frac{\pi}{2}$, the binary value of the sub-carrier is ``1'', otherwise it is ``0'' at the data frame.
% 続けて読み込むパケット長はlength frameから決定する
The number of data frames to read continuously is decided from the second data frame of the packet.
% 1パケット読み込み終わったらCRC-16チェックし、誤り検出されなかったら透かしとして確定する
After reading all data frames of a packet, it is extracted as an annotation if it passes CRC-16 error detection.
