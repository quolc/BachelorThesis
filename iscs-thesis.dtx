% \iffalse meta-comment
%
% 'iscs-thesis' ドキュメントクラス
%
%  東京大学大学院情報理工学系研究科コンピュータ科学専攻
%  / 東京大学理学部情報科学科 の学位論文を組版するための
%  ドキュメントクラスです.
%
%  著作者:  山本 泰宇 (ymmt@is.s.u-tokyo.ac.jp) [原作]
%           八登 崇之 (yato@is.s.u-tokyo.ac.jp) [修正]
%
%  このファイルを pLaTeX2e でコンパイルすると, 詳しい説明
%  の書かれた文書(dvi ファイル)が生成されます.
%  このソフトウェアは誰でも自由に使用および配布できますが,
%  使用した結果に対しては著作者は責任を負わないものとします.
%  このソフトウェアを改変する場合には, 上記の説明文書の中に
%  書かれた注意書きを守ってください.
% \fi
%^^A--------------------------------------------------------
%\DisableCrossrefs
%\CodelineNumbered
%\MakeShortVerb{\|}
%\CheckSum{3478}
%^^A--------------------------------------------------------
%\iffalse  # driver 部は出力しない
%    \begin{macrocode}
%<*driver>
\documentclass[a4j]{jarticle}
\usepackage{doc}
\setcounter{StandardModuleDepth}{1}
\GetFileInfo{iscs-thesis.dtx}
%\OnlyDescription
%\RecordChanges
%--------
\newcommand*{\PKN}[1]{\textsf{#1}}
\newcommand*{\DCN}[1]{\textsc{#1}}
\newcommand*{\OPN}[1]{\texttt{#1}}
\newcommand{\pTeX}{p\kern-.05em\TeX}
\newcommand{\pLaTeX}{p\LaTeX}
\newcommand{\pLaTeXe}{p\LaTeXe}
\newcommand{\Jof}{\leavevmode\lower.5ex\hbox{\rm J}\kern-.17em}
\newcommand{\JTeX}{\Jof\TeX}
\newcommand{\JLaTeX}{\Jof\LaTeX}
\newcommand{\JLaTeXe}{\Jof\LaTeXe}
\newcommand*{\qparag}[1]{\noindent\textbf{#1}\quad}
\newcommand{\SpGlue}{\hspace{0em plus 5em}}
%--------
\begin{document}
\DocInput{iscs-thesis.dtx}
\end{document}
%    \end{macrocode}
%</driver>
%\fi
%^^A========================================================
%\title{ドキュメントクラス \textsf{iscs-thesis} (v1.3)}
%\author{八登 崇之 (yato@is.s.u-tokyo.ac.jp)}
%\date{2009/03/22}
%\maketitle
%^^A--------------------------------------------------------
%\iffalse # Why is it needed?
%<*skip>
%\fi
%
%^^A========================================================
%\section{概説}
%\label{sec:Intro}
%^^A--------------------------------------------------------
%
%本ドキュメントクラス \PKN{iscs-thesis} は,
%東京大学大学院情報理工学系研究科コンピュータ科学専攻
%および東京大学理学部情報科学科の
%学位論文(卒業論文/修士論文/博士論文)を組版するためのものです.
%
%^^A----------------
%\subsection{インストール}
%\label{ssec:Install}
%
%|iscs-thesis.dtx| と |iscs-thesis.ins| のあるディレクトリで
%\begin{quote}\begin{verbatim}
%platex iscs-thesis.ins
%\end{verbatim}\end{quote}
%を実行すると, そのディレクトリに |iscs-thesis.cls| が
%生成されるので, その |iscs-thesis.cls| を \TeX\ が読める
%ディレクトリ(論文のソースファイルのあるディレクトリ等)に
%置いてください.
%なお, |iscs-thesis.cls| が一緒に配布されている場合は,
%第三者による改変が行われていない限り,
%それは上述の方法で生成されるものと
%(漢字コードの差異を除いて)同一です.
%
%\par\noindent
%注意 1: v1.1e から配布する.cls ファイルを JIS エンコーディングに
%しました.
%JIS のファイルは, SJIS または EUC ベースの \pTeX システムでも使えます.
%\par\noindent
%注意 2: このソースには, \DCN{docstrip} の公式のモジュール定義
%はありません.
%(わからない人は気にしないように.)
%
%
%^^A----------------
%\subsection{クラスオプション}
%
%すなわち, 論文のソースの冒頭の
%\begin{quote}\begin{verbatim}
%\documentclass[...]{iscs-thesis}
%\end{verbatim}\end{quote}
%の `|...|' の部分に書くものです.
%\LaTeXe\ の \PKN{report} クラス(欧文用)と概ね同じ
%ですが, 変更点を示しました.
%
%\begin{description}
%\item[論文の種類(追加)]
%  |senior| (卒業論文), |master| (修士論文),
%  |doctor| (博士論文) のいずれか 1 つを
%  必ず指定してください.
%\item[基底フォントサイズ]
%  |10pt|, |11pt|, |12pt| のいずれか.
%  v1.2 から既定値が |11pt| に変更.
%\item[\OPN{interim}]
%  表紙を中間報告(要旨提出)用のものにします.
%\item[Overfull box の設定]
%  何と |draft| (出力する) を既定値にしています.
%  消したい場合は |final| を指定してください.
%\item[\OPN{sloppy}]
%  単語間が空き過ぎになるのを許容して, 行分割が失敗する
%  (その結果行からはみ出して出力される)のを防ぎます.
%  Overfull に対処している暇がない時の応急処置に使えます.
%  (プリアンブルに |\sloppy| を書いたのと同じです.)
%\item[用紙サイズ]
%  |a4paper| (A4 判, 既定値), |letterpaper| (US letter size),
%  |legalpaper| (US legal size) のほかに,
%  新たに |b4paper| (JIS B4 版, 364\,mm $\times$ 257\,mm)
%  を追加しました.
%  (もちろん, 学位論文は A4 判のはずですが.)
%\item[要旨の出力の方法]
%  英文と和文の要旨の間の改ページの制御です.
%  \begin{itemize}
%  \item |splitabst|:\quad 必ず改ページを入れます.
%  \item |nosplitabst|:\quad 改ページを入れません.
%  \item |autosplitabst| (既定値):\quad
%    英文と和文の両方が併せて 1 ページに収まる場合は入れず,
%    その他の場合は入れます.
%  \end{itemize}
%  普通は既定値でいいと思います.
%  英文と和文がともに 1〜1.5 ページの量の場合,
%  既定値(|splitabst| と同じ)では 4 ページになりますが,
%  |nosplitabst| を指定して 3 ページにする方を好むかも
%  しれません.
%\item[前付けのページ番号]
%  表題のページを前付けのページに含めるかどうかを指定します.
%  \begin{itemize}
%  \item |counttitlepage| (既定値):\quad
%    表題のページをページ i とします.
%    表題ページの前に別に表紙がある場合はこの設定が適切です.
%  \item |nocounttitlepage| :\quad
%    表題のページの次の紙をページ i とします.
%    簡易製本で表題ページを表紙として扱う場合はこの設定が適切です.
%  \end{itemize}
%\item[\OPN{simpletitlepage}]
%  博士論文を簡易製本する場合に適応し, 表題ページの体裁を
%  製本時の表紙のもの(表題と氏名のみ)に変更します.
%\item[\OPN{nobindoffset}]
%  v1.3a からページレイアウトを計算する時に「綴じ代」を考慮するように
%  しています.
%  このオプションを指定すると綴じ代がないものと扱います.
%\item[\OPN{english}]
%  表紙および要旨の和文部分を出力しません.
%  (ただし, ソースファイルは和文文字を含むので,
%  必ず \pLaTeX\ を使う必要があります.)
%  本文に和文文字がない限り, できる |.dvi| ファイルは
%  和文フォントを含まないものになります.
%\item[\OPN{prodigal}] (このオプションは v1.3 で廃止された.)
%\item[\OPN{longline}]
%  行の長さを妙に大きくする設定にします.
%  通常は行長は英小文字 80 字分の幅に相当する長さになりますが,
%  代わりに, 左右マージンが紙面横幅の 1/12 の長さになります.
%\item[その他諸々]
%  \PKN{report} と同じく,
%  |twocolumn|, |twoside|, |openright|, |openbib|,
%  |fleqn|, |leqno| が使えます.
%\end{description}
%提出する論文を作る場合は最初の 2 つ(と |final|)を
%指定すれば十分です.
%
%\paragraph{廃止したオプション}
%\PKN{report} にあった次のオプションを廃止しました.
%\begin{description}
%\item[\OPN{a5paper}, \OPN{b5paper}, \OPN{executivepaper}]
%  A4 より小さい紙面では表題のページがうまく組めないので.
%\item[\OPN{landscape}]
%  まさか横置きにする人なんていないでしょう.
%\item[\OPN{titlepage}]
%  表題は常に独立のページに出力されます.
%\end{description}
%
%^^A----------------
%\subsection{テンプレート}
%
%\begin{quote}\footnotesize\begin{verbatim}
%\documentclass[master,12pt]{iscs-thesis}
%  % 論文の種類とフォントサイズをオプションに
%%\usepackage{graphicx}% 必要に応じて
%%\usepackage{mysettings}% 自分用設定
%%-------------------
%\etitle{Title in English}
%\jtitle{和文標題}
%\eauthor{Your Name}
%\jauthor{氏名}
%\esupervisor{Name of Your Supervisor}
%\jsupervisor{指導教官氏名}
%\supervisortitle{Title of Your Supervisor} % Professor, etc.
%\date{February 8, 200X}
%%-------------------
%\begin{document}
%\input{abstract}              % 要旨
%   %\begin{eabstract}...\end{eabstract}
%   %\begin{jabstract}...\end{jabstract}
%\maketitle
%\input{acknowledge}           % 謝辞
%   %\begin{acknowledge}...\end{acknowledge}
%\frontmatter          %% 前付け
%\tableofcontents              % 目次
%%\listoffigures               % 図目次
%%\listoftables                % 表目次
%%-------------------
%\mainmatter           %% 本文
%\if0 書く内容のメモ

・メタデータアプローチはフォーマット依存

\fi

\chapter{Introduction}

\section{Motivation}

\section{Our Research}
        % 1 章
%   %\chapter{Introduction}...
%\include{preliminaries}       % 2 章
%\include{another-section}     % 3 章
%\include{yet-another-section} % 4 章
%\include{conclusion}          % 5 章
%%-------------------
%\bibliographystyle{plain}     % 参考文献
%\bibliography{mybib}          %
%%-------------------
%\end{document}
%\end{verbatim}\end{quote}
%
%^^A----------------
%\subsection{コードを変更する場合の注意}
%\label{ssec:Modify}
%
%\PKN{iscs-thesis} のコード(プログラム)を変更する場合には
%次の 2 つがあります.
%
%\paragraph{全使用者のためになる改良}
%つまりバグ取りや機能拡張などです.
%この場合は,
%\begin{quote}
%|.cls| を直接書き換えるのではなく,
%必ず一度 |.dtx| を書き換えて, \ref{ssec:Install} 節で
%書いたインストール作業により新しい |.cls| を得る
%\end{quote}
%ようにしてください.
%|.cls| ファイルは, 単に |.dtx| の中の(大量にある) |%| で
%始まる行を取り除いたものなので, |.cls| の各行に対応する
%行が必ず存在します.
%それを自分の思うように修正すればよいわけです.
%変更履歴を残した方がいいのは勿論ですが,
%それができない場合でも, 最低限バージョン番号は
%きちんと変更しておきましょう.
%そして必ず |.cls| と一緒に |.dtx| も配布しましょう.
%
%\paragraph{自分専用の設定変更}
%この場合に上と同じ手順をとっても構いません
%(配布はしないでしょうが).
%しかし, 自分専用の設定の場合は,
%修正部分を記したパッケージファイルを作成して
%それを読み込むという方法の方が合理的だと思います.
%
%例えば, 図表のキャプションの字の大きさを |\small| に
%変えたいとしましょう.
%|.cls| ファイルを眺めると,
%次のマクロの定義を変えればいいことが分かります.
%\begin{quote}\small\begin{verbatim}
%\long\def\@makecaption#1#2{% この最初に \small を入れる
%  \vskip\abovecaptionskip
%  \sbox\@tempboxa{#1: #2}%
%  ...(中略)...
%  \vskip\belowcaptionskip}
%\end{verbatim}\end{quote}
%そこで, 次の内容のファイル |mystyle.sty| を作ります.
%(他の定義も加えてあります.)
%\begin{quote}\small\begin{verbatim}
%%% キャプションのフォントを \small にする
%\long\def\@makecaption#1#2{%
%  \small                       % 追加
%  \vskip\abovecaptionskip
%  \sbox\@tempboxa{#1: #2}%
%  \ifdim \wd\@tempboxa >\hsize
%    #1: #2\par
%  \else
%    \global \@minipagefalse
%    \hb@xt@\hsize{\hfil\box\@tempboxa\hfil}%
%  \fi
%  \vskip\belowcaptionskip}
%%% \HUGE: 巨大な字を出す(type1cm 等が必要)
%\newcommand\HUGE{\@setfontsize\HUGE{50}{75}}
%\end{verbatim}\end{quote}
%(注意: |\usepackage| で読み込まれるパッケージの中は\ 
%|\makeatletter| の状態で処理されるので,
%|\makeatletter| する必要はありません.)
%
%そして, 次のようにしてこのファイルを読み込ませれば
%自分専用の設定になります.
%\begin{quote}\small\begin{verbatim}
%\documentclass[senior,12pt]{iscs-thesis}
%\usepackage{mystyle}
%  ...(以下略)...
%\end{verbatim}\end{quote}
%
%もし, 止むを得ず |.cls| ファイルを直接編集することに
%なった場合は, せめて「\TeX\ 社会の掟」だけは守りましょう.
%すなわち
%\begin{quote}
%  ドキュメントクラスの名前(|iscs-thesis|)を他の名前に
%  変更しましょう.
%\end{quote}
%この作業は |.cls| ファイル中の `|iscs-thesis|' の文字列を
%新しいものに単純に(テキストエディタ等で)置換するだけでできますが,
%|\ProvidesClass| の中の情報は自分で適当なものに直してください.
%あとファイル名も変更しましょう.
%たとえ自分からは他人に配布する意図がなかったとしても,
%誰かがサーバに置いてある自分用のファイルを勝手に
%コピーして使うかもしれないので\ldots.
%
%^^A----------------
%\subsection{コマンドリファレンス}
%\label{ssec:Commands}
%
%最後に, このドキュメントクラス特有のコマンドと環境
%についてまとめておきます.
%
%\begin{itemize}
%\item 以下の命令・環境は語句や文章を設定する
%  (|\maketitle| で出力される).
%  |\documentclass| と |\maketitle| の間のどこでも使える.
%  \begin{itemize}
%  \item |\etitle{|\meta{str}|}|:\quad 標題(英文).
%  \item |\jtitle{|\meta{str}|}|:\quad 標題(和文).
%  \item |\eauthor{|\meta{str}|}|:\quad 著者名(英文).
%  \item |\jauthor{|\meta{str}|}|:\quad 著者名(和文).
%  \item |\esupervisor{|\meta{str}|}|:\quad 指導教官名(英文).
%  \item |\jsupervisor{|\meta{str}|}|:\quad 指導教官名(和文).
%  \item |\supervisortitle{|\meta{str}|}|:\quad 
%    指導教官の職名. Professor 等.
%  \item |\supervisortitleline{|\meta{str}|}|:\quad 
%    指導教官の職名の行の全体.
%    |\supervisortitle| で指定した文字列は, \meta{str} の中で\ 
%    |\thesupervisortitle| として参照できる.
%  \item |\date{|\meta{str}|}|:\quad 日付.
%    未設定だとエラーになる.
%    ただし |\today| は使える.
%  \item |eabstract| 環境:\quad 英文要旨.
%  \item |jabstract| 環境:\quad 和文要旨.
%  \end{itemize}
%
%\item |\maketitle|:\quad
%  表紙のページを出力し, 続いて |eabstract|, |jabstract| で
%  設定された要旨を出力する.
%  設定に応じて, 表紙と要旨の間に空白ページが挿入される.
%
%\item |acknowledge| 環境:\quad
%  謝辞の文章を新たなページに出力する.
%
%\item |\switchinterim{|\meta{yes}|}{|\meta{no}|}|:\quad
%|interim| 指定時は \meta{yes}, それ以外は \meta{no} に
%展開される.
%
%\item |\switchenglish{|\meta{yes}|}{|\meta{no}|}|:\quad
%|english| 指定時は \meta{yes}, それ以外は \meta{no} に
%展開される.
%
%\item |\chapterfont{|\meta{cmd1}|}{|\meta{cmd2}|}|:\quad
%番号付(|\chapter|)および
%番号なし(|\chapter*|)の章見出しのフォントをそれぞれ\ 
%\meta{cmd1} および \meta{cmd2} に設定する.
%初期値は両方とも |\LARGE\bfseries|.
%
%\item |\sectionfont{|\meta{cmd1}|}{|\meta{cmd2}|}{|\meta{cmd3}|}|:\quad
%節(|\section|), \SpGlue 小節(|\subsection|), \SpGlue
%小々節(|\subsubsection|)の見出しのフォントをそれぞれ\ 
%\meta{cmd1}, \meta{cmd2}, \meta{cmd3} に設定する.
%初期値は節が |\large\bfseries|,
%小節と小々節が |\normalsize\bfseries|.
%
%\item |\noblankaftertp|:\quad
%  表紙ページ直後の空白ページの出力を抑止する.
%  (|twoside| および |openright| 指定時は無効.)
%
%\end{itemize}
%
%
%^^A========================================================
%\section{変更履歴}
%\label{sec:Changes}
%^^A--------------------------------------------------------
%
%^^A \changes って結局役に立たないな....
%
%\begin{description}
%\item[Version 1.0] [1996/12/22, 山本]
%  \begin{itemize}
%  \item 初期バージョン.
%  \item \JLaTeXe\ 標準の `\PKN{j-report}' クラスを基にしている.
%    学位論文は英語なのになぜ和文用のクラスを用いたのかは不明.
%  \item そのため, 一部の設定(段落下げの量など)が和文用のもの
%    になっているという不具合があった.
%  \item また \JLaTeXe\ の標準クラスオプションファイル
%    (|j-size10.clo| など)を読み込むので,
%    \JLaTeX\ がインストールされていないシステム
%    (最近では \pLaTeX\ が主流なのでこれもよくある)では
%    これらのファイルを別に用意しなければならなかった.
%  \item この版では `\PKN{j-report}' にある全てのクラスオプション
%    が指定できたが, |nottitlepage| 等の実際に使われ得ない
%    ものは実装されていない.
%    実を言えば, 1 つだけ謎のオプションが追加されているのだが\ldots.
%    何を意図したのだろう.
%    (v1.1 では廃止した.)
%  \end{itemize}
%
%\item[Version 1.1] [2005/02/20, 八登]
%  \begin{itemize}
%  \item v1.0 で設定が \PKN{j-report} のままになっていて, かつ,
%    \PKN{j-report} と \PKN{report} (\LaTeXe\ 標準) で異なっている
%    部分については, なるべく \PKN{report} に合わせた.
%    ただし, テキスト領域の大きさや行送りなど, 一部のパラメタ
%    (おもに |j-sizeXX.clo| の前半で設定されているもの)は
%    \PKN{j-report} のままにしている.
%    これによる顕著な変更は次の 2 つ:
%    \begin{itemize}
%    \item 章(|\chapter|)や節(|\section|)等の見出し直後の段落下げ
%      をしなくなった. (欧文ではしないのが普通.)
%      する設定に戻す場合には, \PKN{indentfirst} パッケージを
%      使えばよい.
%    \item 段落下げの量を 1.5\,em (二段組の場合は 1\,em) に
%      変更した.
%      元は 1\,zw だった.
%    \end{itemize}
%  \item クラスオプションについて, 無意味なものを廃止した.
%    またそれによって決して実行されなくなるコードを
%    取り除いた.
%  \item 元々クラスオプションファイル(|j-sizeXX.clo|)になっていた
%    部分を本体に組み込んで, 1 つのファイルだけで使えるようにした.
%  \item クラスファイルを \DCN{docstrip} ソース(|.dtx| ファイル)の
%    形で配布することにした.
%    こうした理由の 1 つはこの版に正統性を持たせるためである.
%  \end{itemize}
%
%\item[Version 1.1a] [2005/02/24, 八登]
%  \begin{itemize}
%  \item `b3' 版の変更を取り入れた.
%    \begin{itemize}
%    \item 表紙(標題)のフォントサイズおよび垂直空きが
%      基底フォントサイズに依らずに一定になるようにした.
%      ただ, |\textwidth| の値が異なるので,
%      完全に同じにはならない.
%    \item 標題が長い時に, 表紙に入るべき内容が 2 ページに
%      分割されてしまう現象を起こりにくくした.
%      とりあえず 7 行(英語と日本語あわせて)までは大丈夫.
%    \item 要旨の中の段落下げの量を,
%      英文(|eabstract|)が 1.5\,em, 和文(|jabstract|)が 1\,zw
%      に修正した.
%    (元はそれぞれ 0\,em と 1\,em.)
%  \item 参考文献リストの見出し(つまり `References') が
%    目次に出るようにした.
%  \end{itemize}
%  \item さらに別の改変版に基づいて次を変更した.
%    \begin{itemize}
%    \item 修士/博士の場合の学位を「理学」から「情報理工学」
%      (Degree of \ldots\ of Information Science and Technology
%      in Computer Science) に変更した.
%      (今まで変更されてなかったの!?)
%      ただし, |gradiss| オプションを指定すると「理学」のままになる.
%      昔の「理学系研究科情報科学専攻」の論文を改めて組版する
%      ためのもの.
%      ちなみに提出先は単に「東京大学大学院」なので変更なし.
%    \item |interim| オプションを設けた.
%      これを指定すると, 表紙が中間報告(要旨提出)のための
%      ものになる,
%  \end{itemize}
%  \item |sloppy| オプションを設けた.
%  \item |senior|, |master|, |doctor| のどれも指定されていないと
%    エラー終了するようにした.
%  \item |draft| を既定値にした. (嫌がらせ.)
%  \item |description| 環境の定義を \PKN{jsarticle} と同様の
%    ものに変更した.
%  \item 和文フォントの明示的な代替設定を行った.
%  \item 日付(|\date|)が設定されていないとエラーが出るようにした.
%  \item その他, エラー処理を強化した.
%  \end{itemize}
%
%\item[Version 1.1b] [2005/02/25, 八登]
%  \begin{itemize}
%  \item |\frontmatter|, |\mainmatter|, |\backmatter|
%    を正式に採用.
%  \item それに伴い, テンプレートを変更した.
%  \item 表紙のレイアウトを調整した.
%    学位論文が共著になるわけがないので |\and| を廃止.
%  \end{itemize}
%
%\item[Version 1.1c] [2005/02/27, 八登]
%  \begin{itemize}
%  \item 要旨の処理(|eabstract| と |jabstract|)の定義を
%    全面的に書き直した.
%    \begin{itemize}
%    \item 従来の処理では
%      要旨環境の中での改ページが禁止されていた.
%      これは「和文と英文の両方が 1 ページに収まらない場合は,
%      別ページに分ける」
%      という機能を実現するためだと思われる,
%      しかし, これだと, 和文だけで 1 ページ分の量を超える
%      場合には, その出力がテキスト領域(あるいは紙面自体)を
%      はみ出してしまう.
%    \item これに対処するために, 要旨の処理方法を変更して,
%      要旨の途中で改ページができるようにした.
%      そして, 前記の機能に対応するため, 事前に 2 つの box の
%      高さの合計を調べて処理を分けている.
%      (詳細は |\ist@showabstract| 命令の説明を参照.
%      この辺りの処理の妥当性については自信がないので, \TeX\ に
%      詳しい方は再検討してください.)
%    \item |interim| 指定の時は, 標題(表紙)と要旨の間に
%      空白のページを置くのを抑止した.
%    \end{itemize}
%  \item |twoside| や |openright| を指定している時には
%    ページ番号の偶奇が保たれるようにしなければならないが,
%    そうなっていなかったために, 奇数/偶数ページの設定が
%    逆転してしまうことがあった.
%    (この現象は, \PKN{report} クラスで |twoside| と |titlepage|
%    を指定して |abstract| 環境を用いた時にも起こる.)
%    この不具合を直して, これらのオプションがきちんと
%    働くようにした.
%    (論文を自分用に印刷する時に両面にする人は多いけど,
%    わざわざ両面用の設定にする人なんていないよな\ldots.)
%  \end{itemize}
%
%\item[Version 1.1d] [2005/03/03, 八登]
%  \begin{itemize}
%  \item 間違った |.cls| ファイルが出力されていたので修正した.
%  \item |splitabst| / |nosplitabst| / |autosplitabst|
%    オプションを追加.
%  \item |prodigal| オプションを追加.
%    レイアウトはまだあまり調整していない.
%  \item |english| オプションを追加.
%  \item 配布する |.cls| ファイルを JIS エンコーディングにしようと
%    して, \texttt{platex --kanji=jis iscs-thesis.ins} とすると,
%    なぜか出てくる |.cls| が EUC になって困った.
%    (\texttt{--kanki} はこのような目的で使用するオプション
%    ではないらしい.)
%  \end{itemize}
%
%\item[Version 1.1e] [2005/12/17, 八登]
%  \begin{itemize}
%  \item 結局, 配布用の |.cls| ファイルは後処理で JIS エンコーディング
%    に変換することにした.
%  \end{itemize}
%
%\item[Version 1.1f] [2005/12/25, 八登]
%  \begin{itemize}
%  \item 指導教官の職名を表す行全体を |\supervisortitleline|
%    でカスタマイズ可能にした.
%    そして, |master|/|doctor| の時の既定値を\ 
%    ``\dots\ of Computer Science'' に変更した.
%  \item |interim| 指定時の表紙で, ``An Interim Report'' の下に
%    日本語で「中間報告」と出るようにした.
%    (これを変更する場合は, |\jinterrimname| を再定義せよ.)
%  \item |\switchinterim|, |\switchenglish| コマンドを新設.
%  \item |\noblankaftertp| コマンドを新設.
%  \item |\maketitle| を |\maketitlepage| と |\makeabstract|
%    に分離する準備を始めている.
%    現時点では, |\maketitle| の処理は(論理的に) v1.1e と同じ.
%  \end{itemize}
%
%\item[Version 1.1g] [2006/06/29, 八登]
%  \begin{itemize}
%  \item |\etitle| の中で |\\| (強制改行) を使うと
%    エラーになっていたのを修正.
%  \end{itemize}
%
%\item[Version 1.2] [2008/12/24]
%
%\item[Version 1.3] [2009/01/22, 八登]
%  \begin{itemize}
%  \item レイアウトを全面的に改訂した.
%    \begin{itemize}
%    \item 時代錯誤的な「ダブルスペース」の要請がなくなったので,
%      行送りを \PKN{report} のものに合わせた.
%    \item 縦方向のマージンを, ヘッダがないという前提で
%      設定するようにした.
%      今の設定でヘッダを使うと上部が窮屈になるので注意.
%    \item 横方向のマージンは, 行の長さが英小文字 75 字に
%      なるように設定した.
%    \end{itemize}
%  \item 基底文字サイズの既定値を v1.2 に合わせて 11pt に変更.
%  \item |prodigal| オプションを廃止.
%  \item |longline| オプションを追加.
%    行をやたらと長くする.
%  \item 表紙のページの内容が常に縦方向にセンタリング
%    されるようにした.
%  \item 博士論文の表紙の体裁を変更.
%  \end{itemize}
%
%\item[Version 1.3a] [2009/03/11, 八登]
%  \begin{itemize}
%  \item 表題ページの後の空白ページを置かないのを既定にした.
%  \item (|no|)|counttitlepage| オプションを追加.
%  \item |simpletitlepage| オプションを追加.
%    博士論文の簡易製本の時の表題ページ(表紙を兼ねる)の体裁を
%    このオプションで指定するようにした.
%  \item ページレイアウトの計算方法を変更した.
%    \begin{itemize}
%    \item 「綴じ」の領域(9\,mm)を考慮することにした.
%    \item |nobindoffset| オプションを追加.
%      これが有効の時は「綴じ」の領域を無視する.
%    \item テキスト領域を紙面サイズの 5/6 に設定した.
%      (ただし |longline| 非設定時は, 行長制限のために
%       横幅はこれより狭くなる.)
%    \item ヘッダ・フッタ領域をテキスト領域から外した.
%      ノンブルはテキスト領域の外側(下側)に配置される.
%    \item マージン幅は左右で $1:1$, 上下で $2:3$ とした.
%    \item |longline| 非設定時の行長制限を 75 字相当から 80 字
%      相当に緩和した.
%    \end{itemize}
%  \end{itemize}
%
%\end{description}
%
%
%^^A--------------------------------------------------------
%\iffalse
%</skip>
%\fi
%\StopEventually{}
%\pagebreak
%^^A--------------------------------------------------------
%
%^^A========================================================
%\section{プログラム}
%\label{sec:Program}
%^^A--------------------------------------------------------
%
%以下の文中で,
%\begin{itemize}
%\item `\PKN{report}' は \LaTeXe\ (v1.4e\,[2001/04/21]) 標準
%  の \PKN{report} クラス
%\item `\PKN{book}' は \LaTeXe\ (v1.4e\,[2001/04/21]) 標準
%  の \PKN{book} クラス
%\item `\PKN{j-report}' は \JLaTeXe\ (v1.4b\,[2000/05/19]) 標準
%  の \PKN{j-report} クラス
%\item `\PKN{jsarticle}' は奥村晴彦氏作成の
%  「\pLaTeXe\ 新ドキュメントクラス」([2004/12/29])の\ 
%  \PKN{jsarticle} クラス
%\end{itemize}
%のことを指す.
%
%^^A----------------
%\subsection{クラスファイルの宣言}
%
%    \begin{macrocode}
%<*!isten>
\NeedsTeXFormat{LaTeX2e}[1999/01/01]
\ProvidesClass{iscs-thesis}
    [2009/03/11 v1.3a
     Dept of IS/CS thesis class]
%    \end{macrocode}
%
%エラー処理のための命令.
%    \begin{macrocode}
\newcommand\ist@classname{iscs-thesis}
\newcommand\ist@ahya{%
  You cannot go any further.\MessageBreak
  Type \space X <return> \space to quit.}
\newcommand*\ist@fatalerror[1]{%
  \ClassError\ist@classname{#1}\ist@ahya
  \batchmode\@@end}% bombout
\newcommand*\ist@error[1]{%
  \ClassError\ist@classname{#1}\@ehc}
\newcommand*\ist@err@invalid[1]{%
  \ist@fatalerror{\string#1 is invalid in this document class}}
\newcommand*\ist@err@notdefd[1]{%
  \ist@error{No \string#1 given}??}
%    \end{macrocode}
%
%^^A----------------
%\subsection{オプションスイッチ}
%
% \begin{macro}{\if@restonecol}
% \begin{macro}{\if@titlepage}
% \begin{macro}{\if@openright}
% \begin{macro}{\if@mainmatter}
%基本的に \PKN{report} と同じ.
%ただし, |titlepage| オプションがないので,
%|\if@titlepage| は常に真となる.
%また, \PKN{book} と同様の |\mainmatter| 等のコマンドの
%ために |\if@mainmatter| を用意する.
%    \begin{macrocode}
\newcommand\@ptsize{}
\newif\if@restonecol
\newif\if@titlepage \@titlepagetrue
\newif\if@openright
\newif\if@mainmatter \@mainmattertrue
%    \end{macrocode}
% \end{macro}
% \end{macro}
% \end{macro}
% \end{macro}
%
%\begin{macro}{\if@seniorthesis}
%\begin{macro}{\if@masterthesis}
%\begin{macro}{\if@doctorthesis}
%どの種類の論文であるかを表すスイッチ.
%必ず丁度 1 つが真になる.
%    \begin{macrocode}
\newif\if@seniorthesis
\newif\if@masterthesis
\newif\if@doctorthesis
%    \end{macrocode}
%\end{macro}
%\end{macro}
%\end{macro}
%
%\begin{macro}{\ifist@interim}
%\begin{macro}{\ifist@gradiss}
%\begin{macro}{\ifist@sloppy}
%\begin{macro}{\ifist@english}
%\begin{macro}{\ifist@blankaftertp}
%\begin{macro}{\ist@splitabst}
%その他のオプションに対するスイッチやマクロ.
%\changes{v1.1a}{2005/02/24}
%  {`ifist@interim 追加.}
%\changes{v1.1a}{2005/02/24}
%  {`ifist@sloppy 追加.}
%\changes{v1.1d}{2005/02/28}
%  {`ifist@prodigal, `ifist@splitabst 追加.}
%\changes{v1.1d}{2005/03/03}
%  {`ifist@english 追加.}
%\changes{v1.1f}{2005/12/25}
%  {`ifist@blankaftertp 追加.}
%\changes{v1.3a}{2009/03/11}
%  {`ifist@blankaftertp の既定値を偽に変更.}
%    \begin{macrocode}
\newif\ifist@interim
\newif\ifist@gradiss
\newif\ifist@sloppy
\newif\ifist@english
\newif\ifist@blankaftertp
\newcommand\ist@splitabst{}
%    \end{macrocode}
%\end{macro}
%\end{macro}
%\end{macro}
%\end{macro}
%\end{macro}
%\end{macro}
%
%\begin{macro}{\ifist@longline}
%\begin{macro}{\ifist@counttitlepage}
%\begin{macro}{\ifist@simpletitlepage}
%v1.3 で追加されたオプションに対するもの.
%\changes{v1.3}{2009/01/22}
%  {`ifist@longline 追加, `ifist@prodigal 廃止.}
%\changes{v1.3a}{2009/02/14}
%  {`ifist@counttitlepage, `ifist@bindoffset 追加.}
%\changes{v1.3a}{2009/03/11}
%  {`ifist@simpletitlepage 追加.}
%    \begin{macrocode}
\newif\ifist@longline
\newif\ifist@counttitlepage
\newif\ifist@bindoffset
\newif\ifist@simpletitlepage
%    \end{macrocode}
%\end{macro}
%\end{macro}
%\end{macro}
%
%\begin{macro}{\bindoffset}
%「綴じ」のために必要な用紙の端の幅。
%\changes{v1.3}{2009/01/22}
%  {`binfoffset 追加,}
%    \begin{macrocode}
\newlength\bindoffset
%    \end{macrocode}
%\end{macro}
%
%^^A----------------
%\subsection{オプションの宣言}
%
%原稿サイズについての変更点は \ref{sec:Intro} 節で述べた通り.
%    \begin{macrocode}
\DeclareOption{a4paper}
   {\setlength\paperheight {297mm}%
    \setlength\paperwidth  {210mm}}
\DeclareOption{b4paper}
   {\setlength\paperheight {364mm}%
    \setlength\paperwidth  {257mm}}
\DeclareOption{letterpaper}
   {\setlength\paperheight {11in}%
    \setlength\paperwidth  {8.5in}}
\DeclareOption{legalpaper}
   {\setlength\paperheight {14in}%
    \setlength\paperwidth  {8.5in}}
%    \end{macrocode}
%
%以下のものは \PKN{report} と同じ.
%    \begin{macrocode}
\DeclareOption{10pt}{\renewcommand\@ptsize{0}}
\DeclareOption{11pt}{\renewcommand\@ptsize{1}}
\DeclareOption{12pt}{\renewcommand\@ptsize{2}}
\DeclareOption{oneside}{\@twosidefalse \@mparswitchfalse}
\DeclareOption{twoside}{\@twosidetrue  \@mparswitchtrue}
\DeclareOption{draft}{\setlength\overfullrule{5pt}}
\DeclareOption{final}{\setlength\overfullrule{0pt}}
\DeclareOption{openright}{\@openrighttrue}
\DeclareOption{openany}{\@openrightfalse}
\DeclareOption{onecolumn}{\@twocolumnfalse}
\DeclareOption{twocolumn}{\@twocolumntrue}
\DeclareOption{leqno}{\input{leqno.clo}}
\DeclareOption{fleqn}{\input{fleqn.clo}}
\DeclareOption{openbib}{%
  \AtEndOfPackage{%
   \renewcommand\@openbib@code{%
      \advance\leftmargin\bibindent
      \itemindent -\bibindent
      \listparindent \itemindent
      \parsep \z@
      }%
   \renewcommand\newblock{\par}}%
}
%    \end{macrocode}
%
%|senior| 等のオプションの処理.
%\changes{v1.1a}{2005/02/24}
%  {interim, sloppy 追加.}
%    \begin{macrocode}
\DeclareOption{senior}%
  {\@seniorthesistrue \@masterthesisfalse \@doctorthesisfalse}
\DeclareOption{master}%
  {\@seniorthesisfalse \@masterthesistrue \@doctorthesisfalse}
\DeclareOption{doctor}%
  {\@seniorthesisfalse \@masterthesisfalse \@doctorthesistrue}
\DeclareOption{interim}{\ist@interimtrue}
\DeclareOption{gradiss}{\ist@gradisstrue}
\DeclareOption{sloppy}{\ist@sloppytrue}
%    \end{macrocode}
%
%v1.1d で追加されたオプションの処理.
%\changes{v1.1d}{2005/02/28}
%  {prodigal, splitabst 追加.}
%\changes{v1.1d}{2005/03/03}
%  {english 追加.}
%    \begin{macrocode}
\DeclareOption{splitabst}{\renewcommand\ist@splitabst{s}}
\DeclareOption{nosplitabst}{\renewcommand\ist@splitabst{n}}
\DeclareOption{autosplitabst}{\renewcommand\ist@splitabst{a}}
\DeclareOption{english}{\ist@englishtrue}
%    \end{macrocode}
%
%v1.3 で追加されたオプションの処理.
%\changes{v1.3}{2009/01/22}
%  {longline 追加, prodigal 廃止.}
%\changes{v1.3a}{2009/02/14}
%  {counttitlepage, nocounttitlepage, nobindoffset 追加.}
%\changes{v1.3a}{2009/03/11}
%  {simpletitlepage 追加.}
%    \begin{macrocode}
\DeclareOption{longline}{\ist@longlinetrue}
\DeclareOption{counttitlepage}{\ist@counttitlepagetrue}
\DeclareOption{nocounttitlepage}{\ist@counttitlepagefalse}
\ist@bindoffsettrue
\DeclareOption{nobindoffset}{\ist@bindoffsetfalse}
\DeclareOption{prodigal}{% now invalid
  \ist@fatalerror{You should not be prodigal in today's world!}}
\DeclareOption{simpletitlepage}{\ist@simpletitlepagetrue}
%    \end{macrocode}
%
%\subsection{オプションの実行}
%
%既定値の設定, およびオプションの処理の実行.
%v1.1a からは |draft| を既定値とする.
%\changes{v1.1a}{2005/02/24}
%  {draft を既定値にする.}
%\changes{v1.1a}{2005/02/24}
%  {senior を既定値にしない.}
%\changes{v1.3}{2009/01/22}
%  {基底フォントサイズの既定値を 10pt から 11pt に変更.}
%    \begin{macrocode}
\ExecuteOptions{a4paper,11pt,oneside,onecolumn,draft,openany,%
            autosplitabst,counttitlepage}
\ProcessOptions
%    \end{macrocode}
%
%|senior|, |master|, |doctor| のどれも指定されていない場合
%はエラー終了する.
%    \begin{macrocode}
\if@seniorthesis\else \if@masterthesis\else
  \if@doctorthesis\else
    \ist@fatalerror{%
      None of `senior', `master', or `doctor'\MessageBreak
      is specified as option}
\fi\fi\fi
%    \end{macrocode}
%
%\begin{macro}{\ifist@carepage}
%|\ifist@carepage| は |twoside| と |openright| のいずれかが
%指定されている場合に真となる.
%これが真の場合には, むやみにページ番号(|\c@page|)を
%リセットすることができない.
%\changes{v1.1c}{2005/02/27}
%  {`ifist@carepage 追加.}
%    \begin{macrocode}
\newif\ifist@carepage
\if@twoside \ist@carepagetrue \fi
\if@openright \ist@carepagetrue \fi
%    \end{macrocode}
%\end{macro}
%
%\begin{macro}{\ist@engine}
%|\ist@engine| は用いている \TeX\ の種類を表す:
%|p| = \pTeX, |j| = \JTeX, |e| = 欧文 \TeX.
%これが |e| の時は, 自動的に |english| モードにする.
%\changes{v1.1d}{2005/03/03}
%  {`ist@engine 追加.}
%    \begin{macrocode}
\newcommand\ist@engine{e}
\@ifundefined{inhibitglue}{}{\renewcommand\ist@engine{p}}
\@ifundefined{jendlinetype}{}{\renewcommand\ist@engine{j}}
\if e\ist@engine \ist@englishtrue \fi
%    \end{macrocode}
%\end{macro}
%
%\begin{macro}{\switchinterim}
%|\switchinterim{|\meta{yes}|}{|\meta{no}|}|: \SpGlue
%|interim| 指定時は \meta{yes}, それ以外は \meta{no} に
%展開される.
%\changes{v1.1f}{2005/12/25}
%  {`switchinterim, `switchenglish 追加.}
%    \begin{macrocode}
\newcommand\switchinterim[2]{%
  \ifist@interim #1\else #2\fi
}
%    \end{macrocode}
%\end{macro}
%\begin{macro}{\switchenglish}
%|\switchenglish{|\meta{yes}|}{|\meta{no}|}|: \SpGlue
%|english| 指定時は \meta{yes}, それ以外は \meta{no} に
%展開される.
%    \begin{macrocode}
\newcommand\switchenglish[2]{%
  \ifist@english #1\else #2\fi
}
%    \end{macrocode}
%\end{macro}
%
%\changes{v1.3a}{2009/03/11}
%  {`blankaftertp を新設.}
%\begin{macro}{\blankaftertp}
%\begin{macro}{\noblankaftertp}
%|\blankaftertp|/|\noblankaftertp|: \SpGlue
%表紙ページ直後の空白ページの挿入を有効/無効にする.
%    \begin{macrocode}
\newcommand\blankaftertp{%
  \ist@blankaftertptrue
}\newcommand\noblankaftertp{%
  \ist@blankaftertpfalse
}
%    \end{macrocode}
%\end{macro}
%\end{macro}
%
%
%\subsection{フォント}
%
%この小節の設定および後の設定の一部は, 元々の \PKN{report} では\ 
%|sizeXX.clo| (\PKN{j-report} では |j-sizeXX.clo|,
%|XX| は基底フォントサイズ) という補助ファイルから読み込んでいたが,
%ここでは, |\@ptsize| の値による条件分岐をして設定を仕分ける
%ことにする.
%こうしても問題はないと思う.
%\changes{v1.3}{2009/01/22}
%  {行送りを全面的に report に合わせる.}
%
%最初に基底フォントサイズオプションが |10pt| の時の設定.
%    \begin{macrocode}
\if0\@ptsize\relax            %--------- 10pt
%    \end{macrocode}
%
%フォントサイズ指定のユーザ命令では,
%同時に行送りの大きさも指定する.
%以下では, \PKN{report} の値をそのまま用いている.
%\par\noindent ※\quad
%v1.1 以前の時代は学位論文の体裁として「ダブルスペース」
%(タイプライタにおいて改行を二重に行う)
%が要請されていた.
%タイプ打ちでない通常の組版においてダブルスペースが何を意味するか
%は微妙な話であるが, v1.1 では和文用(\PKN{j-report})の行送りの
%設定値を全面的に採用していた.\footnote
%{\texttt{10pt} の \texttt{normalsize} での \PKN{j-report} の
%行送りは 16.8\,pt である. 本来の「ダブルスペース」だと 20\,pt
%だから随分違う. これは \PKN{setspace} 等のパッケージを
%参考にした際の作者(八登)の勘違いに起因する.}
%現在は, この時代錯誤的な「ダブルスペース」の要請が削除されて
%いるので, 普通の欧文の行送りに従えばよい.
%
%\begin{macro}{\normalsize}
%\begin{macro}{\small}
%\begin{macro}{\footnotesize}
%|10pt| の場合の設定.
%レイアウト設定を伴うもの.
%    \begin{macrocode}
\renewcommand\normalsize{%
   \@setfontsize\normalsize\@xpt\@xiipt
   \abovedisplayskip 10\p@ \@plus2\p@ \@minus5\p@
   \abovedisplayshortskip \z@ \@plus3\p@
   \belowdisplayshortskip 6\p@ \@plus3\p@ \@minus3\p@
   \belowdisplayskip \abovedisplayskip
   \let\@listi\@listI}
\normalsize
\newcommand\small{%
   \@setfontsize\small\@ixpt{11}%
   \abovedisplayskip 8.5\p@ \@plus3\p@ \@minus4\p@
   \abovedisplayshortskip \z@ \@plus2\p@
   \belowdisplayshortskip 4\p@ \@plus2\p@ \@minus2\p@
   \def\@listi{\leftmargin\leftmargini
               \topsep 4\p@ \@plus2\p@ \@minus2\p@
               \parsep 2\p@ \@plus\p@ \@minus\p@
               \itemsep \parsep}%
   \belowdisplayskip \abovedisplayskip
}
\newcommand\footnotesize{%
   \@setfontsize\footnotesize\@viiipt{9.5}%
   \abovedisplayskip 6\p@ \@plus2\p@ \@minus4\p@
   \abovedisplayshortskip \z@ \@plus\p@
   \belowdisplayshortskip 3\p@ \@plus\p@ \@minus2\p@
   \def\@listi{\leftmargin\leftmargini
               \topsep 3\p@ \@plus\p@ \@minus\p@
               \parsep 2\p@ \@plus\p@ \@minus\p@
               \itemsep \parsep}%
   \belowdisplayskip \abovedisplayskip
}
%    \end{macrocode}
%\end{macro}
%\end{macro}
%\end{macro}
%\begin{macro}{\scriptsize}
%\begin{macro}{\tiny}
%\begin{macro}{\large}
%\begin{macro}{\Large}
%\begin{macro}{\huge}
%\begin{macro}{\Huge}
%伴わないもの.
%    \begin{macrocode}
\newcommand\scriptsize{\@setfontsize\scriptsize\@viipt\@viiipt}
\newcommand\tiny{\@setfontsize\tiny\@vpt\@vipt}
\newcommand\large{\@setfontsize\large\@xiipt{14}}
\newcommand\Large{\@setfontsize\Large\@xivpt{18}}
\newcommand\LARGE{\@setfontsize\LARGE\@xviipt{22}}
\newcommand\huge{\@setfontsize\huge\@xxpt{25}}
\newcommand\Huge{\@setfontsize\Huge\@xxvpt{30}}
%    \end{macrocode}
%\end{macro}
%\end{macro}
%\end{macro}
%\end{macro}
%\end{macro}
%\end{macro}
%
%\begin{macro}{\textwidth}
%\begin{macro}{\topskip}
%\begin{macro}{\marginparsep}
%\begin{macro}{\marginparpush}
%\noindent ($*$)
%ここで基底サイズに依存する他の長さ変数を設定する.
%
%|\textwidth| は本文領域の幅で, 既定の設定(学位論文用の設定)で
%はここで設定された値がそのまま使われる.
%欧文の組版の場合, 行の長さは大体英小文字 65 字分が理想と
%され, 長くても 75 字を超えてはならないとされる.
%ただし, 読む人が慣れている場合に限り 80 文字まで可とされる
%\footnote{\PKN{KOMA-script} クラスのドキュメント参照.
%  計算機科学関連の書籍では行長が長いものが多く散見される.}.
%以上の事情を勘案した結果,
%このクラスでは, なるべく版面を大きくとれるように,
%行長を 80 文字相当の長さにした.
%算出方法は, \PKN{memoir} クラスの方法を適用した場合の Computer Modern
%の「65 字相当幅」の 80/65 倍を超えない最大の 12\,pt (= 1\,pc) の
%整数倍とした.
%\changes{v1.3}{2009/01/22}
%  {`textwidth の既定値を 75 字幅とする.}
%\changes{v1.3a}{2009/02/11}
%  {計算を間違っていたので修正.}
%\iffalse
%   F    Laz     L65     L    N      F=基底フォントサイズ
%  10pt 127.58  293.94  360  79.6    Laz= abc..xyz の幅
%  11pt 139.70  318.69  384  78.3    L65= 65字相当幅
%  12pt 149.88  339.46  408  78.1%   L= 行長 / N= 文字数
%  L65 = 2.042 Laz + 33.41 ; L = floor(L65*80/65/12)*12
%\fi
%    \begin{macrocode}
\setlength\textwidth{360\p@}
\setlength\topskip{10\p@}
\setlength\marginparsep{11\p@}
\setlength\marginparpush{5\p@}
%    \end{macrocode}
%\end{macro}
%\end{macro}
%\end{macro}
%\end{macro}
%以上で |10pt| の場合の設定は終わり.
%
%続いて |11pt| の場合.
%説明は |10pt| の時と同じなので省略.
%    \begin{macrocode}
\else\if1\@ptsize\relax       %--------- 11pt
\renewcommand\normalsize{%
   \@setfontsize\normalsize\@xipt{13.6}%
   \abovedisplayskip 11\p@ \@plus3\p@ \@minus6\p@
   \abovedisplayshortskip \z@ \@plus3\p@
   \belowdisplayshortskip 6.5\p@ \@plus3.5\p@ \@minus3\p@
   \belowdisplayskip \abovedisplayskip
   \let\@listi\@listI}
\normalsize
\newcommand\small{%
   \@setfontsize\small\@xpt\@xiipt
   \abovedisplayskip 10\p@ \@plus2\p@ \@minus5\p@
   \abovedisplayshortskip \z@ \@plus3\p@
   \belowdisplayshortskip 6\p@ \@plus3\p@ \@minus3\p@
   \def\@listi{\leftmargin\leftmargini
               \topsep 6\p@ \@plus2\p@ \@minus2\p@
               \parsep 3\p@ \@plus2\p@ \@minus\p@
               \itemsep \parsep}%
   \belowdisplayskip \abovedisplayskip
}
\newcommand\footnotesize{%
   \@setfontsize\footnotesize\@ixpt{11}%
   \abovedisplayskip 8\p@ \@plus2\p@ \@minus4\p@
   \abovedisplayshortskip \z@ \@plus\p@
   \belowdisplayshortskip 4\p@ \@plus2\p@ \@minus2\p@
   \def\@listi{\leftmargin\leftmargini
               \topsep 4\p@ \@plus2\p@ \@minus2\p@
               \parsep 2\p@ \@plus\p@ \@minus\p@
               \itemsep \parsep}%
   \belowdisplayskip \abovedisplayskip
}
\newcommand\scriptsize{\@setfontsize\scriptsize\@viiipt{9.5}}
\newcommand\tiny{\@setfontsize\tiny\@vipt\@viipt}
\newcommand\large{\@setfontsize\large\@xiipt{14}}
\newcommand\Large{\@setfontsize\Large\@xivpt{18}}
\newcommand\LARGE{\@setfontsize\LARGE\@xviipt{22}}
\newcommand\huge{\@setfontsize\huge\@xxpt{25}}
\newcommand\Huge{\@setfontsize\Huge\@xxvpt{30}}
\setlength\textwidth{384\p@}
\setlength\topskip{11\p@}
\setlength\marginparsep{10\p@}
\setlength\marginparpush{5\p@}
%    \end{macrocode}
%最後に |12pt| の場合.
%    \begin{macrocode}
\else                         %--------- 12pt
\renewcommand\normalsize{%
   \@setfontsize\normalsize\@xiipt{14.5}%
   \abovedisplayskip 12\p@ \@plus3\p@ \@minus7\p@
   \abovedisplayshortskip \z@ \@plus3\p@
   \belowdisplayshortskip 6.5\p@ \@plus3.5\p@ \@minus3\p@
   \belowdisplayskip \abovedisplayskip
   \let\@listi\@listI}
\normalsize
\newcommand\small{%
   \@setfontsize\small\@xipt{13.6}%
   \abovedisplayskip 11\p@ \@plus3\p@ \@minus6\p@
   \abovedisplayshortskip \z@ \@plus3\p@
   \belowdisplayshortskip 6.5\p@ \@plus3.5\p@ \@minus3\p@
   \def\@listi{\leftmargin\leftmargini
               \topsep 9\p@ \@plus3\p@ \@minus5\p@
               \parsep 4.5\p@ \@plus2\p@ \@minus\p@
               \itemsep \parsep}%
   \belowdisplayskip \abovedisplayskip
}
\newcommand\footnotesize{%
   \@setfontsize\footnotesize\@xpt\@xiipt
   \abovedisplayskip 10\p@ \@plus2\p@ \@minus5\p@
   \abovedisplayshortskip \z@ \@plus3\p@
   \belowdisplayshortskip 6\p@ \@plus3\p@ \@minus3\p@
   \def\@listi{\leftmargin\leftmargini
               \topsep 6\p@ \@plus2\p@ \@minus2\p@
               \parsep 3\p@ \@plus2\p@ \@minus\p@
               \itemsep \parsep}%
   \belowdisplayskip \abovedisplayskip
}
\newcommand\scriptsize{\@setfontsize\scriptsize\@viiipt{9.5}}
\newcommand\tiny{\@setfontsize\tiny\@vipt\@viipt}
\newcommand\large{\@setfontsize\large\@xivpt{18}}
\newcommand\Large{\@setfontsize\Large\@xviipt{22}}
\newcommand\LARGE{\@setfontsize\LARGE\@xxpt{25}}
\newcommand\huge{\@setfontsize\huge\@xxvpt{30}}
\let\Huge=\huge
\setlength\textwidth{408\p@}
\setlength\topskip{12\p@}
\setlength\marginparsep{10\p@}
\setlength\marginparpush{7\p@}
\fi\fi                        %---------
%    \end{macrocode}
%以上で基底フォントサイズ依存部分は一旦終了.
%
%\changes{v1.1a}{2005/02/24}
%  {`isttitlesize 追加.}
%\begin{macro}{\isttitlesize}
%タイトル用のフォントサイズ.
%基底フォントサイズに依らないようにする.
%内容は |10ot| の |\Large| と同じ.
%    \begin{macrocode}
\newcommand\isttitlesize{\@setfontsize\isttitlesize\@xivpt{25.2}}
%    \end{macrocode}
%\end{macro}
%
%\qparag{和文フォントの代替設定}
%和文フォントについての「代替されました」の警告メッセージを
%止めるために, 明示的な代替設定をしておく.
%\changes{v1.1a}{2005/02/24}
%  {和文フォントの明示的な代替の設定.}
%    \begin{macrocode}
\if p\ist@engine\relax
\DeclareFontShape{JY1}{mc}{m}{it}{<->ssub*mc/m/n}{}
\DeclareFontShape{JT1}{mc}{m}{it}{<->ssub*mc/m/n}{}
\DeclareFontShape{JY1}{mc}{m}{sc}{<->ssub*mc/m/n}{}
\DeclareFontShape{JT1}{mc}{m}{sc}{<->ssub*mc/m/n}{}
\DeclareFontShape{JY1}{mc}{m}{sl}{<->ssub*mc/m/n}{}
\DeclareFontShape{JT1}{mc}{m}{sl}{<->ssub*mc/m/n}{}
\DeclareFontShape{JY1}{mc}{bx}{it}{<->ssub*gt/m/n}{}
\DeclareFontShape{JT1}{mc}{bx}{it}{<->ssub*gt/m/n}{}
\DeclareFontShape{JY1}{mc}{bx}{sc}{<->ssub*gt/m/n}{}
\DeclareFontShape{JT1}{mc}{bx}{sc}{<->ssub*gt/m/n}{}
\DeclareFontShape{JY1}{mc}{bx}{sl}{<->ssub*gt/m/n}{}
\DeclareFontShape{JT1}{mc}{bx}{sl}{<->ssub*gt/m/n}{}
\DeclareFontShape{JY1}{gt}{m}{it}{<->ssub*gt/m/n}{}
\DeclareFontShape{JT1}{gt}{m}{it}{<->ssub*gt/m/n}{}
\DeclareFontShape{JY1}{gt}{m}{sc}{<->ssub*gt/m/n}{}
\DeclareFontShape{JT1}{gt}{m}{sc}{<->ssub*gt/m/n}{}
\DeclareFontShape{JY1}{gt}{m}{sl}{<->ssub*gt/m/n}{}
\DeclareFontShape{JT1}{gt}{m}{sl}{<->ssub*gt/m/n}{}
\DeclareFontShape{JY1}{gt}{bx}{it}{<->ssub*gt/m/n}{}
\DeclareFontShape{JT1}{gt}{bx}{it}{<->ssub*gt/m/n}{}
\DeclareFontShape{JY1}{gt}{bx}{sc}{<->ssub*gt/m/n}{}
\DeclareFontShape{JT1}{gt}{bx}{sc}{<->ssub*gt/m/n}{}
\DeclareFontShape{JY1}{gt}{bx}{sl}{<->ssub*gt/m/n}{}
\DeclareFontShape{JT1}{gt}{bx}{sl}{<->ssub*gt/m/n}{}
\fi
%    \end{macrocode}
%
%
%^^A----------------
%\subsection{文書レイアウト}
%
%\begin{macro}{\bindoffset}
%「綴じ」に必要な幅の設定.
%    \begin{macrocode}
\ifist@bindoffset
  \setlength{\bindoffset}{9mm}
\else
  \setlength{\bindoffset}{0pt}
\fi
%    \end{macrocode}
%\end{macro}
%
%\begin{macro}{\lineskip}
%\begin{macro}{\normallineskip}
%\begin{macro}{\baselinestretch}
%\begin{macro}{\parskip}
%\qparag{段落}
%これらは \PKN{report} のまま.
%    \begin{macrocode}
\setlength\lineskip{1\p@}
\setlength\normallineskip{1\p@}
\renewcommand\baselinestretch{}
%    \end{macrocode}
%\end{macro}
%\end{macro}
%\end{macro}
%\end{macro}

%\begin{macro}{\parindent}
%段落下げは 1.5\,em (二段組では 1\,em) に統一した.
%これは \PKN{report} の値とほぼ同じ.
%v1.0 では \PKN{j-report} のままの 1\,zw となっていたが,
%これは明らかに不合理.
%    \begin{macrocode}
\setlength\parskip{0\p@ \@plus \p@}
\if@twocolumn
  \setlength\parindent{1em}
\else
  \setlength\parindent{1.5em}
\fi
%    \end{macrocode}
%\end{macro}
%
%\begin{macro}{\smallskipamount}
%\begin{macro}{\medskipamount}
%\begin{macro}{\bigskipamount}
%\begin{macro}{\@lowpenalty}
%\begin{macro}{\@medpenalty}
%\begin{macro}{\@highpenalty}
%\PKN{report} のまま.
%    \begin{macrocode}
\setlength\smallskipamount{3\p@ \@plus 1\p@ \@minus 1\p@}
\setlength\medskipamount{6\p@ \@plus 2\p@ \@minus 2\p@}
\setlength\bigskipamount{12\p@ \@plus 4\p@ \@minus 4\p@}
\@lowpenalty   51
\@medpenalty  151
\@highpenalty 301
%    \end{macrocode}
%\end{macro}
%\end{macro}
%\end{macro}
%\end{macro}
%\end{macro}
%\end{macro}
%
%\begin{macro}{\headheight}
%\begin{macro}{\headsep}
%\begin{macro}{\footskip}
%\begin{macro}{\maxdepth}
%\qparag{縦方向の空き}
%|\headsep| を小さくした以外は \PKN{report} のまま.
%|\topskip| は ($*$) で設定済.
%\changes{v1.3}{2009/01/22}
%  {`headsep を変更.}
%    \begin{macrocode}
\setlength\headheight{12\p@}
\setlength\headsep   {12\p@}
% \topskip is already set
\setlength\footskip{30\p@}
\setlength\maxdepth{.5\topskip}
%    \end{macrocode}
%\end{macro}
%\end{macro}
%\end{macro}
%\end{macro}
%
%\begin{macro}{\textwidth}
%\qparag{テキスト領域の大きさ}
%幅の設定.
%この時点で |\textwidth| には ($*$) で設定した値が入っている.
%二段組(|twocolumn|)または |longline| 設定時は, 
%マージンを綴じを除いた紙面の幅の 1/6 とする.
%すなわち |\textwidth|
%を (|\paperwidth| $-$ |\bindoffset|) $\times$ 5/6 ($\dagger$)とする.
%それ以外の場合は, ($*$) と ($\dagger$) のうち小さい方とする.
%\changes{v1.3}{2009/01/22}
%  {`textwidth の設定方法を変更.}
%    \begin{macrocode}
\setlength\@tempdima{\paperwidth}
\addtolength\@tempdima{-\bindoffset}
\setlength\@tempdima{.833333\@tempdima}
\if@twocolumn
  \setlength\textwidth\@tempdima
\else\ifist@longline
  \setlength\textwidth\@tempdima
\else\ifdim\textwidth>\@tempdima\relax
  \setlength\textwidth\@tempdima
\fi\fi\fi
\@settopoint\textwidth
%    \end{macrocode}
%\end{macro}
%
%\begin{macro}{\textheight}
%テキスト領域の高さの設定.
%用紙の高さの 1/6 をマージンとする.
%\PKN{report} では, ここでヘッダ・フッタの領域として 1.5\,in
%を確保しているが, このクラスではヘッダ・フッタの領域をとらない.
%つまり, テキスト領域の外側に配置される.
%\changes{v1.3}{2009/01/22}
%  {`textheight の設定方法を変更.}
%\changes{v1.3a}{2009/02/14}
%  {`textheight の設定方法を変更.}
%    \begin{macrocode}
\setlength\@tempdima{.833333\paperheight}
\divide\@tempdima\baselineskip
\@tempcnta=\@tempdima
\setlength\textheight{\@tempcnta\baselineskip}
\addtolength\textheight{\topskip}
%    \end{macrocode}
%\end{macro}
%
%\begin{macro}{\marginparsep}
%\qparag{マージン}
%\PKN{report} のまま.
%|\marginparpush| は ($*$) で設定済み.
%    \begin{macrocode}
\if@twocolumn
 \setlength\marginparsep {10\p@}
\else
% \marginparsep is unchanged
\fi
% \marginparpush is already set
%    \end{macrocode}
%\end{macro}
%
%\begin{macro}{\oddsidemargin}
%\begin{macro}{\evensidemargin}
%\begin{macro}{\marginparwidth}
%これらの値は |\textwidth| から算出される.
%    \begin{macrocode}
\if@twoside
  \setlength\@tempdima        {\paperwidth}
  \addtolength\@tempdima      {-\bindoffset}
  \addtolength\@tempdima      {-\textwidth}
  \setlength\oddsidemargin    {.333333\@tempdima}
  \addtolength\oddsidemargin  {-1in}
  \addtolength\oddsidemargin  {\bindoffset}
  \setlength\evensidemargin   {.666667\@tempdima}
  \addtolength\evensidemargin {-1in}
  \setlength\marginparwidth   {.666667\@tempdima}
  \addtolength\marginparwidth {-\marginparsep}
  \addtolength\marginparwidth {-0.4in}
\else
  \setlength\@tempdima        {\paperwidth}
  \addtolength\@tempdima      {-\bindoffset}
  \addtolength\@tempdima      {-\textwidth}
  \setlength\oddsidemargin    {.5\@tempdima}
  \addtolength\oddsidemargin  {-1in}
  \addtolength\oddsidemargin  {\bindoffset}
  \setlength\marginparwidth   {.5\@tempdima}
  \addtolength\marginparwidth {-\marginparsep}
  \addtolength\marginparwidth {-0.4in}
  \addtolength\marginparwidth {-.4in}
  \setlength\evensidemargin   {\oddsidemargin}
\fi
\ifdim \marginparwidth >2in
   \setlength\marginparwidth{2in}
\fi
\@settopoint\oddsidemargin
\@settopoint\marginparwidth
\@settopoint\evensidemargin
%    \end{macrocode}
%\end{macro}
%\end{macro}
%\end{macro}
%
%\begin{macro}{\topmargin}
%これらの値は |\textheight| から算出される.
%\PKN{report} とは異なり, 中央合わせの際にヘッダ・フッタ部分を
%含めないようにしている.
%\changes{v1.3}{2009/01/22}
%  {`topmargin の算出方法を変更.}
%\changes{v1.3a}{2009/02/14}
%  {`topmargin の算出方法を変更.}
%    \begin{macrocode}
\setlength\@tempdima{\paperheight}
\addtolength\@tempdima{-\textheight}
\setlength\topmargin{.4\@tempdima}
\addtolength\topmargin{-1in}
\addtolength\topmargin{-\headheight}
\addtolength\topmargin{-\headsep}
\@settopoint\topmargin
%    \end{macrocode}
%\end{macro}
%
%\qparag{脚注}
%|\footnotesep|, |\skip\footins| の設定は後回し.
%
%^^A---------------
%\subsection{フロートの設定}
%
%\qparag{許容範囲}
%これは \PKN{jsarticle} に合わせるように変更した.
%元よりもフロートが入りやすくなるはず.
%    \begin{macrocode}
\setcounter{topnumber}{2}
\renewcommand\topfraction{.8}
\setcounter{bottomnumber}{1}
\renewcommand\bottomfraction{.8}
\setcounter{totalnumber}{3}
\renewcommand\textfraction{.1}
\renewcommand\floatpagefraction{.8}
\setcounter{dbltopnumber}{2}
\renewcommand\dbltopfraction{.8}
\renewcommand\dblfloatpagefraction{.8}
%    \end{macrocode}
%
%残りの設定は基底フォントサイズに依存するので後回し.
%
%^^A----------------
%\subsection{ページスタイル}
%すなわちヘッダ・フッタの設定.
%
%|headings| スタイルの設定は,
%v1.0 では \PKN{j-report} と同じであったが,
%今では \PKN{report} と同じ.
%    \begin{macrocode}
\if@twoside
  \def\ps@headings{%
      \let\@oddfoot\@empty\let\@evenfoot\@empty
      \def\@evenhead{\thepage\hfil\slshape\leftmark}%
      \def\@oddhead{{\slshape\rightmark}\hfil\thepage}%
      \let\@mkboth\markboth
    \def\chaptermark##1{%
      \markboth {\MakeUppercase{%
        \ifnum \c@secnumdepth >\m@ne
          \if@mainmatter
            \@chapapp\ \thechapter. \ %
          \fi
        \fi
        ##1}}{}}%
    \def\sectionmark##1{%
      \markright {\MakeUppercase{%
        \ifnum \c@secnumdepth >\z@
          \thesection. \ %
        \fi
        ##1}}}}
\else
  \def\ps@headings{%
    \let\@oddfoot\@empty
    \def\@oddhead{{\slshape\rightmark}\hfil\thepage}%
    \let\@mkboth\markboth
    \def\chaptermark##1{%
      \markright {\MakeUppercase{%
        \ifnum \c@secnumdepth >\m@ne
          \if@mainmatter
            \@chapapp\ \thechapter. \ %
          \fi
        \fi
        ##1}}}}
\fi
\def\ps@myheadings{%
    \let\@oddfoot\@empty\let\@evenfoot\@empty
    \def\@evenhead{\thepage\hfil\slshape\leftmark}%
    \def\@oddhead{{\slshape\rightmark}\hfil\thepage}%
    \let\@mkboth\@gobbletwo
    \let\chaptermark\@gobble
    \let\sectionmark\@gobble
    }
%    \end{macrocode}
%
%\begin{macro}{\ist@saveps}
%\begin{macro}{\ist@restoreps}
%|\ist@saveps|\,/\,|\ist@restoreps| は現在のページスタイルを
%退避/復帰する.
%\changes{v1.1c}{2005/02/27}
%  {`ist@saveps, `ist@restoreps 追加.}
%    \begin{macrocode}
\newcommand\ist@saveps{%
  \let\ist@mkboth\@mkboth
  \let\ist@oddhead\@oddhead\let\ist@oddfoot\@oddfoot
  \let\ist@evenhead\@evenhead\let\ist@evenfoot\@evenfoot
}
\newcommand\ist@restoreps{%
  \let\@mkboth\ist@mkboth
  \let\@oddhead\ist@oddhead\let\@oddfoot\ist@oddfoot
  \let\@evenhead\ist@evenhead\let\@evenfoot\ist@evenfoot
}
%    \end{macrocode}
%\end{macro}
%\end{macro}
%
%^^A----------------
%\subsection{文書マークアップ}
%
%\qparag{タイトル}
%すなわち学位論文の表紙のページ.
%\PKN{report} で |titlepage| オプションを指定したのと同様に,
%常に独立のページに出力される,
%v1.1a〜1.1c で全面的な見直しを行った.
%\changes{v1.1a}{2005/02/24}
%  {`maketitle についていろいろと変更.}
%\changes{v1.1b}{2005/02/25}
%  {`maketitle についてさらにと変更.}
%\changes{v1.1c}{2005/02/27}
%  {`maketitle の処理の大部分を下請け命令に移した.}
%\changes{v1.1c}{2005/02/27}
%  {要旨の出力方法を大幅に変更.}
%\changes{v1.1f}{2005/12/25}
%  {表紙と要旨の命令の分離. `maketitle はそれらを呼ぶだけ.}
%\begin{macro}{\maketitle}
%|\maketitlepage| による表紙出力の直後に\ 
%|\makeabstract| による要旨出力を行う
%(v1.1f よりこの 2 命令を新設).
%表紙と要旨の間には通常は空白のページが置かれる(v1.0 と同様)が,
%|interim| 指定の場合は置かれない.
%^^A#v1.1c において, 実際に表紙を出力する部分を |\ist@maketitle| に,
%^^A#要旨を出力する部分を |\ist@showabstract| に移した.
%^^A#(設定変更を容易にするため.)
%\changes{v1.3}{2009/02/14}
%  {ページ番号の数え方を変更.}
%    \begin{macrocode}
\newcommand\maketitle{%
  \pagenumbering{roman}%
  \maketitlepage
  \ist@putblankpage
  \ifist@counttitlepage\else \setcounter{page}\@ne \fi
  \makeabstract
}
%    \end{macrocode}
%\end{macro}
%
%\begin{environment}{ist@titlepage}
%ページ番号をリセットしない |titlepage| 環境.
%    \begin{macrocode}
\newenvironment{ist@titlepage}
 {\ifist@carepage \cleardoublepage \fi
  \if@twocolumn \@restonecoltrue\onecolumn
  \else \@restonecolfalse\newpage
  \fi       	
  \thispagestyle{empty}}
 {\if@restonecol \twocolumn
  \else \newpage \fi}
%    \end{macrocode}
%\end{environment}
%
%\begin{macro}{\ist@putblankpage}
%空白のページを出力するための処理.
%(v1.0 でなぜ空白ページを置くのかは未だ不明.)
%    \begin{macrocode}
\newcommand\ist@putblankpage{%
  \ifist@interim \ist@blankaftertpfalse \fi
  \ifist@carepage \ist@blankaftertptrue \fi
  \ifist@blankaftertp
    \null\vfil \thispagestyle{empty}% make an empty page
    \newpage
  \fi
}
%    \end{macrocode}
%\end{macro}
%\begin{macro}{\maketitlepage}
%|\etitle|, |\date| 等を設定した後に |\maketitlepage| を実行すると,
%論文の表紙が出力される.
%\changes{v1.1f}{2005/12/25}
%  {`maketitlepage 追加.}
%    \begin{macrocode}
\newcommand\maketitlepage{%
  \ist@maketitle
  \ist@maketitle@post
}
%    \end{macrocode}
%\end{macro}
%\begin{macro}{\makeabstract}
%|eabstract| および |jabstract| 環境
%を用いて入力された要旨(英文および和文)が出力される.
%\changes{v1.1f}{2005/12/25}
%  {`makeabstract 追加.}
%    \begin{macrocode}
\newcommand\makeabstract{%
  \ist@showabstract
  \ist@showabstract@post
}
%    \end{macrocode}
%\end{macro}
%
%\begin{macro}{\ist@maketitle}
%実際に表紙のページを出力する命令.
%学位論文が共著になるわけがないので,
%|\and| を廃止して定義を単純にした.
%1 つのブロック内の行送りが常に |\isttitlesize| で設定した
%ものになるようにした.
%今の設定では, タイトルは 12 行(英文・和文あわせて)まで書ける.
%    \begin{macrocode}
\newcommand\ist@maketitle{\begin{ist@titlepage}%
  \let\footnotesize\small
  \let\footnoterule\relax
  \let \footnote \thanks
  \null\vskip-100\p@\@plus1fill\null
  \centering\isttitlesize
    {\ist@hookcr\MakeUppercase{\@etitle}}\par
    {\@jtitle}\par
    \vskip 20\p@
    by\par
    \vskip 10\p@
    {\@eauthor\\\@jauthor}\par
    \vskip 30\p@
    \ifist@interim
      {\einterimname\\\jinterimname}\par
    \else
      {\ethesisname\\\jthesisname}\par
    \fi
    \vskip 80\p@
    {\ist@submittedtoblock}\par
    \vskip 20\p@
    {Thesis Supervisor: \@esupervisor \quad \@jsupervisor\\
     \@supervisortitleline}\par
  \vskip-\footskip
  \vskip-100\p@\@plus1fill\null
  \end{ist@titlepage}%
  \setcounter{footnote}{0}%
}
%    \end{macrocode}
%\end{macro}
%\begin{macro}{\ist@hookcr}
%|\etitle| で |\\|(強制改行)を使うとエラーになることへの対処.
%\changes{v1.1g}{2006/06/29}
%  {`ist@hookcr 等を追加.}
%    \begin{macrocode}
\def\ist@hookcr{%
  \let\ist@curcr\\\def\\{\protect\ist@curcr}}
%    \end{macrocode}
%\end{macro}
%
%\begin{macro}{\ist@showabstract}
%実際に要旨を出力する命令.
%
%\noindent ※ 要旨の処理について:\quad
%\label{anc:abstract}
%従来の処理では, まず |ebastract|, |jabstract| 環境で
%各内容を box register に代入して, |\maketitle| において
%その register の中身を出力するという方法をとっていた.
%しかし, その際に中に入れる box として |minipage| 環境を
%中に含んだ |\hbox| を用いていたので, その結果,
%要旨環境の中での改ページが禁止されていた.
%これは「和文と英文の両方が 1 ページに収まらない場合は,
%別ページに分ける」
%という処理を実現するためだと思われる,
%しかし, これだと, 和文だけで 1 ページ分の量を超える
%場合には, その出力がテキスト領域(さらに多いと紙面自体)を
%はみ出してしまう.
%
%これに対処するために, 用いる box を |\vbox| にして,
%さらに, |\unvbox| で出力することで, 要旨の途中で
%改ページができるようにした.
%そして, 要旨が長い時に別ページにする機能に対応するため,
%事前に 2 つの box の高さの合計を調べて処理を分けている.
%(この処理の妥当性については自信がないので, \TeX\ に
%詳しい方は再検討してください.)
%    \begin{macrocode}
\newcommand\ist@showabstract{%
%    \end{macrocode}
%英文と和文の要旨の間に入る垂直空きの量.
%    \begin{macrocode}
  \setlength{\@tempskipb}{36\p@\@minus24\p@}
%    \end{macrocode}
%|autosplitabst| 指定時は,
%(英文要旨の縦幅) + (和文要旨の縦幅) + (挿入する空きの自然長)
%が |\textheight| より大きいか小さいかで処理を分ける.
%大きい場合は, 設定を |splitabst| にする.
%    \begin{macrocode}
  \if a\ist@splitabst \relax
    \setlength\@tempdima{\@tempskipb}%
    \addtolength\@tempdima{\ht\eabstractbox}%
    \addtolength\@tempdima{\dp\eabstractbox}%
    \addtolength\@tempdima{\ht\jabstractbox}%
    \addtolength\@tempdima{\dp\jabstractbox}%
    \ifdim \@tempdima>\textheight
      \renewcommand\ist@splitabst{s}%
    \fi
  \fi
%    \end{macrocode}
%|autosplitabst| でかつ要旨が小さい場合の処理:
%従来通り, |titlepage| 環境を用いて, 両方の要旨を出力する.
%必ず 1 ページに収まるはず.
%(こちらの方が後述の方法よりバグが少ないと思われるので,
%この場合を特別扱いしている.
%本来は, 後述の場合で処理してかまわない.)
%    \begin{macrocode}
  \if a\ist@splitabst \relax
    \begin{ist@titlepage}%
      \unvbox\eabstractbox
      \vskip\@tempskipb
      \unvbox\jabstractbox
    \end{ist@titlepage}%
%    \end{macrocode}
%残りの場合の処理:
%要旨が 3 ページ以上になる場合には, ページスタイルの一時的な変更
%(|empty| に変える)を |titlepage| に任せるという方法が使えない.
%(2 ページならば, |\end{titlepage| の後で |\thispagestyle{empty}|
%をすればよい.)
%仕方がないので, 現在のページスタイルを退避/復帰する命令
%(|\ist@saveps|\,/\,|\ist@restoreps|)を用意して対処した.
%この点を除くと, 前処理・後処理は |titlepage| 環境のそれと同じ.
%|splitabst| 設定時(または |autosplitabst| で要旨が大きい時)は\ 
%2 つの要旨の出力の間で改ページし,
%|nosplitabst| 設定時は 2 つの要旨の間に |\@tempskipb| の
%空きを入れる.
%    \begin{macrocode}
  \else
    \ifist@carepage \cleardoublepage \fi
    \if@twocolumn \@restonecoltrue\onecolumn
    \else         \@restonecolfalse\newpage
    \fi
    \ist@saveps \pagestyle{empty}%
    \unvbox\eabstractbox
    \if s\ist@splitabst\relax \newpage
    \else  \vskip\@tempskipb
    \fi
    \unvbox\jabstractbox
    \if@restonecol\twocolumn \else \newpage \fi
    \ist@restoreps
  \fi
}
%    \end{macrocode}
%\end{macro}
%
%\begin{macro}{\ist@submittedtoblock}
%``Submitted to \ldots'' で始まる文言の内容.
%    \begin{macrocode}
\newcommand\ist@submittedtoblock{%
  Submitted to\\\@submittedto\\
  \ifist@interim\else on \@date\\\fi
  in Partial Fulfillment of the Requirements\\
  for \@degreename
}
\if@seniorthesis
  \newcommand\@submittedto{%
    the Department of Information Science\\
    the Faculty of Science, the University of Tokyo}
  \newcommand\@degreename{%
    the Degree of \thesisgrade \ of Science}
\else
  \newcommand\@submittedto{%
    the Graduate School of the University of Tokyo}
  \ifist@gradiss
    \newcommand\@degreename{%
      the Degree of \thesisgrade \ of Science\\
      in Information Science}
  \else
    \newcommand\@degreename{%
      the Degree of \thesisgrade\ 
      of Information Science and Technology\\
      in Computer Science}
  \fi
\fi
%    \end{macrocode}
%\end{macro}
%
%
%\begin{macro}{\ist@maketitle@post}
%\begin{macro}{\ist@showabstract@post}
%用済みのマクロを消して記憶領域を空ける.
%この処理は今では必要ないのかもしれない.
%    \begin{macrocode}
\newcommand\ist@maketitle@post{%
  \global\let\thanks\relax
  \global\let\@thanks\@empty
  \global\let\@jauthor\@empty
  \global\let\@eauthor\@empty
  \global\let\@date\@empty
  \global\let\@jtitle\@empty
  \global\let\@etitle\@empty
  \global\let\@jsupervisor\@empty
  \global\let\@esupervisor\@empty
  \global\let\@supervisortitle\@empty
  \global\let\@submittedto\@empty
  \global\let\@degree\@empty
  \global\let\ist@submittedtoblock\@empty
  \global\let\einterimname\@empty
  \global\let\jinterimname\@empty
  \global\let\ethesisname\@empty
  \global\let\jthesisname\@empty
  \global\let\thesisgrade\@empty
  \global\let\jtitle\relax
  \global\let\etitle\relax
  \global\let\jauthor\relax
  \global\let\eauthor\relax
  \global\let\jsupervisor\relax
  \global\let\esupervisor\relax
  \global\let\supervisortitle\relax
  \global\let\date\relax
  %
  \global\let\maketitle\relax
  \global\let\maketitlepage\relax
  \global\let\ist@maketitle\relax
  \global\let\ist@maketitle@post\relax
}
\newcommand\ist@showabstract@post{%
  \global\let\makeabstract\relax
  \global\let\ist@putblankpage\relax
  \global\let\ist@showabstract\relax
  \global\let\ist@showabstract@post\relax
}
%    \end{macrocode}
%\end{macro}
%\end{macro}
%
%\qparag{博士論文の表紙}
%博士論文を簡易製本する場合は, 表題ページがそのまま表紙になる
%ので, これを規定の形式に合わせる必要がある.
%|simpletitlepage| オプションでこれを行える.
%    \begin{macrocode}
\ifist@simpletitlepage
\renewcommand\ist@maketitle{\begin{ist@titlepage}%
  \let\footnotesize\small
  \let\footnoterule\relax
  \let \footnote \thanks
  \null\vskip 40\p@\null
  \centering\isttitlesize
    {\@etitle}\par
    \vskip 10\p@
    {\ist@jparen\@jtitle}\par
    \vfill
    \@jauthor\par
  \vskip 10\p@
  \end{ist@titlepage}%
  \setcounter{footnote}{0}%
}
\fi
%    \end{macrocode}
%
%\qparag{\OPN{english} 設定時の表紙}
%|english| 設定時の |\ist@maketitle| と |\ist@showabstract|.
%    \begin{macrocode}
\ifist@english                %--------- english
\renewcommand\ist@maketitle{\begin{ist@titlepage}%
  \let\footnotesize\small
  \let\footnoterule\relax
  \let \footnote \thanks
  \null\vskip-100\p@\@plus1fill\null
  \centering\isttitlesize
    {\ist@hookcr\MakeUppercase{\@etitle}}\par
    \vskip 20\p@
    by\par
    \vskip 10\p@
    {\@eauthor}\par
    \vskip 30\p@
    \ifist@interim
      {\einterimname}\par
    \else
      {\ethesisname}\par
    \fi
    \vskip 80\p@
    {\ist@submittedtoblock}\par
    \vskip 20\p@
    {Thesis Supervisor: \@esupervisor\\
     \@supervisortitleline}\par
  \vskip-\footskip
  \vskip-100\p@\@plus1fill\null
  \end{ist@titlepage}%
  \setcounter{footnote}{0}%
}
\renewcommand\ist@showabstract{%
  \ifist@carepage \cleardoublepage \fi
  \if@twocolumn \@restonecoltrue\onecolumn
  \else         \@restonecolfalse\newpage
  \fi
  \ist@saveps \pagestyle{empty}%
  \unvbox\eabstractbox
  \if@restonecol\twocolumn \else \newpage \fi
  \ist@restoreps
}
\fi                           %---------
%    \end{macrocode}
%
%\qparag{節見出し}
%カウンタ定義などの準備の部分.
%\PKN{report} のまま.
%    \begin{macrocode}
\newcommand*\chaptermark[1]{}
\setcounter{secnumdepth}{2}
\newcounter {part}
\newcounter {chapter}
\newcounter {section}[chapter]
\newcounter {subsection}[section]
\newcounter {subsubsection}[subsection]
\newcounter {paragraph}[subsubsection]
\newcounter {subparagraph}[paragraph]
\renewcommand \thepart {\@Roman\c@part}
\renewcommand \thechapter {\@arabic\c@chapter}
\renewcommand \thesection {\thechapter.\@arabic\c@section}
\renewcommand\thesubsection   {\thesection.\@arabic\c@subsection}
\renewcommand\thesubsubsection{\thesubsection .\@arabic\c@subsubsection}
\renewcommand\theparagraph    {\thesubsubsection.\@arabic\c@paragraph}
\renewcommand\thesubparagraph {\theparagraph.\@arabic\c@subparagraph}
\newcommand\@chapapp{\chaptername}
%    \end{macrocode}
%
%
%\begin{macro}{\frontmatter}
%\begin{macro}{\mainmatter}
%\begin{macro}{\backmatter}
%\PKN{book} で使える「前付け・本文・後付け」の制御を
%取り入れてみた.
%\changes{v1.1b}{2005/02/25}
%  {`mainmatter 等を追加.}
%    \begin{macrocode}
\newcommand\frontmatter{%
  \ist@clearpage
  \@mainmatterfalse}
\newcommand\mainmatter{%
  \ist@clearpage
  \@mainmattertrue
  \pagenumbering{arabic}}
\newcommand\backmatter{%
  \if@openright
    \cleardoublepage
  \else
    \clearpage
  \fi
  \@mainmatterfalse}
%    \end{macrocode}
%\end{macro}
%\end{macro}
%\end{macro}
%
%\begin{macro}{\ist@clearpage}
%|\ist@clearpage| は |twoside| と |openright| のいずれかが
%指定されていれば |\cleardoublepage|,
%そうでなければ |\clearpage| を行う.
%\changes{v1.1b}{2005/02/25}
%  {`ist@clearpage 追加.}
%\changes{v1.1c}{2005/02/27}
%  {`ist@clearpage の定義で `ifist@carepage を用いる.}
%    \begin{macrocode}
\newcommand\ist@clearpage{%
  \ifist@carepage \cleardoublepage \else \clearpage \fi}
%    \end{macrocode}
%\end{macro}
%
%部(part)の見出し.
%    \begin{macrocode}
\newcommand\part{%
  \if@openright
    \cleardoublepage
  \else
    \clearpage
  \fi
  \thispagestyle{plain}%
  \if@twocolumn
    \onecolumn
    \@tempswatrue
  \else
    \@tempswafalse
  \fi
  \null\vfil
  \secdef\@part\@spart}

\def\@part[#1]#2{%
    \ifnum \c@secnumdepth >-2\relax
      \refstepcounter{part}%
      \addcontentsline{toc}{part}{\thepart\hspace{1em}#1}%
    \else
      \addcontentsline{toc}{part}{#1}%
    \fi
    \markboth{}{}%
    {\centering
     \interlinepenalty \@M
     \normalfont
     \ifnum \c@secnumdepth >-2\relax
       \huge\bfseries \partname\nobreakspace\thepart
       \par
       \vskip 20\p@
     \fi
     \Huge \bfseries #2\par}%
    \@endpart}
\def\@spart#1{%
    {\centering
     \interlinepenalty \@M
     \normalfont
     \Huge \bfseries #1\par}%
    \@endpart}
\def\@endpart{\vfil\newpage
              \if@twoside
               \if@openright
                \null
                \thispagestyle{empty}%
                \newpage
               \fi
              \fi
              \if@tempswa
                \twocolumn
              \fi}
%    \end{macrocode}
%
%章(chapter)の見出し.
%v1.0 から少し修正して \PKN{report} と同じにした.
%ただし, 見出しの字の大きさは, \PKN{report} の |\huge| ではなく
%\PKN{j-report} と同じ |\LARGE| である.
%ここのフォント設定は \PKN{j-report} では\ 
%|\chapn@font|, |\chapt@font| というマクロになっていて,
%後述の |\chapterfont| という命令でこれらの中身が
%変えられるようになっている.
%この方式もそのまま引き継いでいる.
%    \begin{macrocode}
\newcommand\chapter{\if@openright\cleardoublepage\else\clearpage\fi
                    \thispagestyle{plain}%
                    \global\@topnum\z@
                    \@afterindentfalse
                    \secdef\@chapter\@schapter}
\def\@chapter[#1]#2{\ifnum \c@secnumdepth >\m@ne
                       \if@mainmatter
                         \refstepcounter{chapter}%
                         \typeout{\@chapapp\space\thechapter.}%
                         \addcontentsline{toc}{chapter}%
                                   {\protect\numberline{\thechapter}#1}%
                       \else
                         \addcontentsline{toc}{chapter}{#1}%
                       \fi
                    \else
                      \addcontentsline{toc}{chapter}{#1}%
                    \fi
                    \chaptermark{#1}%
                    \addtocontents{lof}{\protect\addvspace{10\p@}}%
                    \addtocontents{lot}{\protect\addvspace{10\p@}}%
                    \if@twocolumn
                      \@topnewpage[\@makechapterhead{#2}]%
                    \else
                      \@makechapterhead{#2}%
                      \@afterheading
                    \fi}
\def\@makechapterhead#1{%
  \vspace*{50\p@}%
  {\parindent \z@ \raggedright \normalfont
    \ifnum \c@secnumdepth >\m@ne
      \if@mainmatter
        \chapn@font \@chapapp\space \thechapter
        \par\nobreak
        \vskip 20\p@
      \fi
    \fi
    \interlinepenalty\@M
    \chapt@font #1\par\nobreak
    \vskip 40\p@
  }}
\def\@schapter#1{\if@twocolumn
                   \@topnewpage[\@makeschapterhead{#1}]%
                 \else
                   \@makeschapterhead{#1}%
                   \@afterheading
                 \fi}
\def\@makeschapterhead#1{%
  \vspace*{50\p@}%
  {\parindent \z@ \raggedright
    \normalfont
    \interlinepenalty\@M
    \chapt@font #1\par\nobreak
    \vskip 40\p@
  }}
%    \end{macrocode}
%
%\begin{macro}{\chapterfont}
%|\chapterfont{|\meta{cmd1}|}{|\meta{cmd2}|}|: \SpGlue
%番号付(|\chapter|)および
%番号なし(|\chapter*|)の章見出しのフォントをそれぞれ\ 
%\meta{cmd1} および \meta{cmd2} に設定する.
%    \begin{macrocode}
\newcommand*\chapterfont[2]{%
   \gdef\chapn@font{#1}\gdef\chapt@font{#2}}
%    \end{macrocode}
%\end{macro}
%初期値はともに |\LARGE\bfseries|.
%    \begin{macrocode}
\chapterfont{\LARGE\bfseries}{\LARGE\bfseries}
%    \end{macrocode}
%
%節(section)以下の見出し.
%\PKN{report} (欧文) では |\section| 等の直後の段落下げをしないのに
%対して, \PKN{j-report} ではする.
%元の \PKN{is-thesis} (v1.0) ではするように設定されていたが,
%おそらく欧文ではしないのが普通だと思われるので,
%しない設定に変更した.
%(|\@startsection| の第 4 引数を負にすると「しない」になる.)
%また, 節, 小節, 小々節の見出しの字の大きさも両者で異なり,
%前述の章と同様にこれも \PKN{j-report} ではカスタマイズ可能
%となっている.
%これについては \PKN{j-report} を引き継ぐ.
%    \begin{macrocode}
\newcommand\section{\@startsection {section}{1}{\z@}%
                                   {-3.5ex \@plus -1ex \@minus -.2ex}%
                                   {2.3ex \@plus.2ex}%
                                   {\normalfont\sec@font}}
\newcommand\subsection{\@startsection{subsection}{2}{\z@}%
                                     {-3.25ex\@plus -1ex \@minus -.2ex}%
                                     {1.5ex \@plus .2ex}%
                                     {\normalfont\ssec@font}}
\newcommand\subsubsection{\@startsection{subsubsection}{3}{\z@}%
                                     {-3.25ex\@plus -1ex \@minus -.2ex}%
                                     {1.5ex \@plus .2ex}%
                                     {\normalfont\sssec@font}}
\newcommand\paragraph{\@startsection{paragraph}{4}{\z@}%
                                    {3.25ex \@plus1ex \@minus.2ex}%
                                    {-1em}%
                                    {\normalfont\normalsize\bfseries}}
\newcommand\subparagraph{\@startsection{subparagraph}{5}{\parindent}%
                                       {3.25ex \@plus1ex \@minus .2ex}%
                                       {-1em}%
                                      {\normalfont\normalsize\bfseries}}
%    \end{macrocode}
%
%\begin{macro}{\sectionfont}
%|\sectionfont{|\meta{cmd1}|}{|\meta{cmd2}|}{|\meta{cmd3}|}|: \SpGlue
%節(|\section|), \SpGlue 小節(|\subsection|), \SpGlue
%小々節(|\subsubsection|)の見出しのフォントをそれぞれ\ 
%\meta{cmd1}, \meta{cmd2}, \meta{cmd3} に設定する.
%    \begin{macrocode}
\newcommand*\sectionfont[3]{%
   \gdef\sec@font{#1}\gdef\ssec@font{#2}\gdef\sssec@font{#3}}
%    \end{macrocode}
%\end{macro}
%初期値は節が |\large\bfseries|,
%小節と小々節が |\normalsize\bfseries|.
%なお, \PKN{report} ではサイズが順に\ 
%|\Large|, |\large|, |\normalsize| となっていた.
%    \begin{macrocode}
\sectionfont{\large\bfseries}{\normalsize\bfseries}{\normalsize\bfseries}
%    \end{macrocode}
%
%^^A----------------
%\subsection{リスト}
%
%この小節中の全ての設定は \PKN{report}のまま.
%    \begin{macrocode}
\if@twocolumn
  \setlength\leftmargini  {2em}
\else
  \setlength\leftmargini  {2.5em}
\fi
\leftmargin  \leftmargini
\setlength\leftmarginii  {2.2em}
\setlength\leftmarginiii {1.87em}
\setlength\leftmarginiv  {1.7em}
\if@twocolumn
  \setlength\leftmarginv  {.5em}
  \setlength\leftmarginvi {.5em}
\else
  \setlength\leftmarginv  {1em}
  \setlength\leftmarginvi {1em}
\fi
\setlength  \labelsep  {.5em}
\setlength  \labelwidth{\leftmargini}
\addtolength\labelwidth{-\labelsep}
\@beginparpenalty -\@lowpenalty
\@endparpenalty   -\@lowpenalty
\@itempenalty     -\@lowpenalty
%    \end{macrocode}
%
%ここより 2 回目(で最後)の
%基底フォントサイズ依存部分をはじめる.
%まず |10pt| から.
%    \begin{macrocode}
\if0\@ptsize\relax            %--------- 10pt
%    \end{macrocode}
%
%まずは落穂拾い.
%脚注関係の設定.
%    \begin{macrocode}
\setlength\footnotesep{6.65\p@}
\setlength{\skip\footins}{9\p@ \@plus 4\p@ \@minus 2\p@}
%    \end{macrocode}
%
%フロート関係の設定.
%    \begin{macrocode}
\setlength\floatsep    {12\p@ \@plus 2\p@ \@minus 2\p@}
\setlength\textfloatsep{20\p@ \@plus 2\p@ \@minus 4\p@}
\setlength\intextsep   {12\p@ \@plus 2\p@ \@minus 2\p@}
\setlength\dblfloatsep    {12\p@ \@plus 2\p@ \@minus 2\p@}
\setlength\dbltextfloatsep{20\p@ \@plus 2\p@ \@minus 4\p@}
\setlength\@fptop{0\p@ \@plus 1fil}
\setlength\@fpsep{8\p@ \@plus 2fil}
\setlength\@fpbot{0\p@ \@plus 1fil}
\setlength\@dblfptop{0\p@ \@plus 1fil}
\setlength\@dblfpsep{8\p@ \@plus 2fil}
\setlength\@dblfpbot{0\p@ \@plus 1fil}
\setlength\partopsep{2\p@ \@plus 1\p@ \@minus 1\p@}
%    \end{macrocode}
%リストの設定に戻る.
%    \begin{macrocode}
\def\@listi{\leftmargin\leftmargini
            \parsep 4\p@ \@plus2\p@ \@minus\p@
            \topsep 8\p@ \@plus2\p@ \@minus4\p@
            \itemsep4\p@ \@plus2\p@ \@minus\p@}
\let\@listI\@listi
\@listi
\def\@listii {\leftmargin\leftmarginii
              \labelwidth\leftmarginii
              \advance\labelwidth-\labelsep
              \topsep    4\p@ \@plus2\p@ \@minus\p@
              \parsep    2\p@ \@plus\p@  \@minus\p@
              \itemsep   \parsep}
\def\@listiii{\leftmargin\leftmarginiii
              \labelwidth\leftmarginiii
              \advance\labelwidth-\labelsep
              \topsep    2\p@ \@plus\p@\@minus\p@
              \parsep    \z@
              \partopsep \p@ \@plus\z@ \@minus\p@
              \itemsep   \topsep}
%    \end{macrocode}
%以上で |10ot| の場合は終わり.
%
%続いて |11pt| の場合.
%    \begin{macrocode}
\else\if1\@ptsize\relax       %--------- 11pt
\setlength\footnotesep{7.7\p@}
\setlength{\skip\footins}{10\p@ \@plus 4\p@ \@minus 2\p@}
\setlength\floatsep    {12\p@ \@plus 2\p@ \@minus 2\p@}
\setlength\textfloatsep{20\p@ \@plus 2\p@ \@minus 4\p@}
\setlength\intextsep   {12\p@ \@plus 2\p@ \@minus 2\p@}
\setlength\dblfloatsep    {12\p@ \@plus 2\p@ \@minus 2\p@}
\setlength\dbltextfloatsep{20\p@ \@plus 2\p@ \@minus 4\p@}
\setlength\@fptop{0\p@ \@plus 1fil}
\setlength\@fpsep{8\p@ \@plus 2fil}
\setlength\@fpbot{0\p@ \@plus 1fil}
\setlength\@dblfptop{0\p@ \@plus 1fil}
\setlength\@dblfpsep{8\p@ \@plus 2fil}
\setlength\@dblfpbot{0\p@ \@plus 1fil}
\setlength\partopsep{3\p@ \@plus 1\p@ \@minus 1\p@}
\def\@listi{\leftmargin\leftmargini
            \parsep 4.5\p@ \@plus2\p@ \@minus\p@
            \topsep 9\p@   \@plus3\p@ \@minus5\p@
            \itemsep4.5\p@ \@plus2\p@ \@minus\p@}
\let\@listI\@listi
\@listi
\def\@listii {\leftmargin\leftmarginii
              \labelwidth\leftmarginii
              \advance\labelwidth-\labelsep
              \topsep    4.5\p@ \@plus2\p@ \@minus\p@
              \parsep    2\p@   \@plus\p@  \@minus\p@
              \itemsep   \parsep}
\def\@listiii{\leftmargin\leftmarginiii
              \labelwidth\leftmarginiii
              \advance\labelwidth-\labelsep
              \topsep    2\p@ \@plus\p@\@minus\p@
              \parsep    \z@
              \partopsep \p@ \@plus\z@ \@minus\p@
              \itemsep   \topsep}
%    \end{macrocode}
%
%続いて |12pt| の場合.
%    \begin{macrocode}
\else                         %--------- 12pt
\setlength\footnotesep{8.4\p@}
\setlength{\skip\footins}{10.8\p@ \@plus 4\p@ \@minus 2\p@}
\setlength\floatsep    {12\p@ \@plus 2\p@ \@minus 4\p@}
\setlength\textfloatsep{20\p@ \@plus 2\p@ \@minus 4\p@}
\setlength\intextsep   {14\p@ \@plus 4\p@ \@minus 4\p@}
\setlength\dblfloatsep    {14\p@ \@plus 2\p@ \@minus 4\p@}
\setlength\dbltextfloatsep{20\p@ \@plus 2\p@ \@minus 4\p@}
\setlength\@fptop{0\p@ \@plus 1fil}
\setlength\@fpsep{10\p@ \@plus 2fil}
\setlength\@fpbot{0\p@ \@plus 1fil}
\setlength\@dblfptop{0\p@ \@plus 1fil}
\setlength\@dblfpsep{10\p@ \@plus 2fil}
\setlength\@dblfpbot{0\p@ \@plus 1fil}
\setlength\partopsep{3\p@ \@plus 2\p@ \@minus 2\p@}
\def\@listi{\leftmargin\leftmargini
            \parsep 5\p@  \@plus2.5\p@ \@minus\p@
            \topsep 10\p@ \@plus4\p@   \@minus6\p@
            \itemsep5\p@  \@plus2.5\p@ \@minus\p@}
\let\@listI\@listi
\@listi
\def\@listii {\leftmargin\leftmarginii
              \labelwidth\leftmarginii
              \advance\labelwidth-\labelsep
              \topsep    5\p@   \@plus2.5\p@ \@minus\p@
              \parsep    2.5\p@ \@plus\p@    \@minus\p@
              \itemsep   \parsep}
\def\@listiii{\leftmargin\leftmarginiii
              \labelwidth\leftmarginiii
              \advance\labelwidth-\labelsep
              \topsep    2.5\p@\@plus\p@\@minus\p@
              \parsep    \z@
              \partopsep \p@ \@plus\z@ \@minus\p@
              \itemsep   \topsep}
\fi\fi                        %---------
%    \end{macrocode}
%以上で基底フォントサイズ依存部分は終了.
%
%残りのリスト関係の設定.
%    \begin{macrocode}
\def\@listiv {\leftmargin\leftmarginiv
              \labelwidth\leftmarginiv
              \advance\labelwidth-\labelsep}
\def\@listv  {\leftmargin\leftmarginv
              \labelwidth\leftmarginv
              \advance\labelwidth-\labelsep}
\def\@listvi {\leftmargin\leftmarginvi
              \labelwidth\leftmarginvi
              \advance\labelwidth-\labelsep}
\renewcommand\theenumi{\@arabic\c@enumi}
\renewcommand\theenumii{\@alph\c@enumii}
\renewcommand\theenumiii{\@roman\c@enumiii}
\renewcommand\theenumiv{\@Alph\c@enumiv}
\newcommand\labelenumi{\theenumi.}
\newcommand\labelenumii{(\theenumii)}
\newcommand\labelenumiii{\theenumiii.}
\newcommand\labelenumiv{\theenumiv.}
\renewcommand\p@enumii{\theenumi}
\renewcommand\p@enumiii{\theenumi(\theenumii)}
\renewcommand\p@enumiv{\p@enumiii\theenumiii}
\newcommand\labelitemi{\textbullet}
\newcommand\labelitemii{\normalfont\bfseries \textendash}
\newcommand\labelitemiii{\textasteriskcentered}
\newcommand\labelitemiv{\textperiodcentered}
%    \end{macrocode}
%
%\begin{environment}{description}
%|description| の定義は \PKN{jsarticle} のそれに準じる.
%ただし |\labelsep| は (1\,zw でなくて) 1\,em とする.
%(|\descriptionlabel| の定義方法が異なるので注意せよ.)
%\changes{v1.1a}{2005/02/24}
%  {description の定義を変更.}
%    \begin{macrocode}
\newenvironment{description}
               {\list{}{\labelwidth\leftmargin \labelsep1em%
                        \advance\labelwidth-\labelsep
                        \let\makelabel\descriptionlabel}}
               {\endlist}
\newcommand*\descriptionlabel[1]{\normalfont\bfseries #1\hfil}
%    \end{macrocode}
%\end{environment}
%
%^^A----------------
%\subsection{新しい環境の定義}
%
%\qparag{謝辞}
%    \begin{macrocode}
\newenvironment{acknowledge}
               {\begin{titlepage}
                \vspace*{50\p@}%
                  {\parindent \z@ \raggedright \normalfont
                   \interlinepenalty\@M
                   \chapt@font Acknowledgements\par\nobreak
                   \vskip 40\p@}%
               }
               {\end{titlepage}}
%    \end{macrocode}
%
%\qparag{要旨}
%v1.0 では段落下げの量は, 英文が 0\,em, 和文が 1\,em という
%訳の分からない値になっていたが,
%v1.1a でそれぞれ 1.5\,em と 1\,zw に変更した.
%\changes{v1.1a}{2005/02/24}
%  {要旨環境中の段落下げの量の変更.}
%\changes{v1.1c}{2005/02/27}
%  {要旨出力方法の変更に合わせて要旨環境の定義を修正.}
%
%\noindent ※\quad
%v1.1c において全面的に見直した.
%詳細は \pageref{anc:abstract} ページ参照.
%\changes{v1.3}{2009/01/22}
%  {欧文行送りの変更に伴い, jabstract に `baselinestretch 設定を追加.}
%    \begin{macrocode}
\newsavebox{\eabstractbox}%
\newsavebox{\jabstractbox}%
\newenvironment{eabstract}%
    {\global\setbox\eabstractbox\vbox\bgroup
     \everypar{}% cancel \@nodocument
     \@beginparpenalty\@lowpenalty \small
     \setlength{\parindent}{1.5em}%
     \begin{center}%
       \bfseries\MakeUppercase{\eabstractname}%
       \@endparpenalty\@M
     \end{center}\par}%
    {\par\egroup}
\newenvironment{jabstract}%
    {\global\setbox\jabstractbox\vbox\bgroup
     \everypar{}%
     \renewcommand{\baselinestretch}{1.4}%
     \@beginparpenalty\@lowpenalty \small
     \setlength{\parindent}{1zw}%
     \begin{center}%
       \bfseries \jabstractname
       \@endparpenalty\@M
     \end{center}\par}%
    {\par\egroup}
%    \end{macrocode}
%|english| 設定時の |jabstract| 環境.
%    \begin{macrocode}
\ifist@english
\renewenvironment{jabstract}%
    {\global\setbox\jabstractbox\vbox\bgroup\everypar{}}
    {\par\egroup\global\setbox\jabstractbox\box\voidb@x}
\fi
%    \end{macrocode}
%
%\qparag{韻文}
%論文には関係ないと思うなかれ.
%    \begin{macrocode}
\newenvironment{verse}
               {\let\\\@centercr
                \list{}{\itemsep      \z@
                        \itemindent   -1.5em%
                        \listparindent\itemindent
                        \rightmargin  \leftmargin
                        \advance\leftmargin 1.5em}%
                \item\relax}
               {\endlist}
%    \end{macrocode}
%
%\qparag{引用}
%    \begin{macrocode}
\newenvironment{quotation}
               {\list{}{\listparindent 1.5em%
                        \itemindent    \listparindent
                        \rightmargin   \leftmargin
                        \parsep        \z@ \@plus\p@}%
                \item\relax}
               {\endlist}
\newenvironment{quote}
               {\list{}{\rightmargin\leftmargin}%
                \item\relax}
               {\endlist}
%    \end{macrocode}
%
%\qparag{Titlepage}
%\changes{v1.1c}{2005/02/27}
%  {titlepage 環境の変更: carepage 時の対策.}
%    \begin{macrocode}
\newenvironment{titlepage}
    {\ifist@carepage \cleardoublepage \fi
      \if@twocolumn
        \@restonecoltrue\onecolumn
      \else
        \@restonecolfalse\newpage
      \fi
      \thispagestyle{empty}%
%     \setcounter{page}\@ne
    }%
    {\if@restonecol\twocolumn \else \newpage \fi
%    \ifist@carepage\else \setcounter{page}\@ne \fi
    }
%    \end{macrocode}
%
%\qparag{付録}
%これは環境じゃないけど.
%    \begin{macrocode}
\newcommand\appendix{\par
  \setcounter{chapter}{0}%
  \setcounter{section}{0}%
  \gdef\@chapapp{\appendixname}%
  \gdef\thechapter{\@Alph\c@chapter}}
%    \end{macrocode}
%
%^^A----------------
%\subsection{既存の環境のパラメタ設定}
%
%全て \PKN{report} のまま.
%    \begin{macrocode}
\setlength\arraycolsep{5\p@}
\setlength\tabcolsep{6\p@}
\setlength\arrayrulewidth{.4\p@}
\setlength\doublerulesep{2\p@}
\setlength\tabbingsep{\labelsep}
\skip\@mpfootins = \skip\footins
\setlength\fboxsep{3\p@}
\setlength\fboxrule{.4\p@}
\@addtoreset {equation}{chapter}
\renewcommand\theequation
  {\ifnum \c@chapter>\z@ \thechapter.\fi \@arabic\c@equation}
%    \end{macrocode}
%
%^^A----------------
%\subsection{フロートの定義}
%
%今では \PKN{report} と完全に同じにしている.
%(v1.0 ではこの定義を \PKN{j-report} と同じにして,
%別のパラメタ設定で \PKN{report} に合わせていた.)
%    \begin{macrocode}
\newcounter{figure}[chapter]
\renewcommand \thefigure
     {\ifnum \c@chapter>\z@ \thechapter.\fi \@arabic\c@figure}
\def\fps@figure{tbp}
\def\ftype@figure{1}
\def\ext@figure{lof}
\def\fnum@figure{\figurename\nobreakspace\thefigure}
\newenvironment{figure}
               {\@float{figure}}
               {\end@float}
\newenvironment{figure*}
               {\@dblfloat{figure}}
               {\end@dblfloat}
\newcounter{table}[chapter]
\renewcommand \thetable
     {\ifnum \c@chapter>\z@ \thechapter.\fi \@arabic\c@table}
\def\fps@table{tbp}
\def\ftype@table{2}
\def\ext@table{lot}
\def\fnum@table{\tablename\nobreakspace\thetable}
\newenvironment{table}
               {\@float{table}}
               {\end@float}
\newenvironment{table*}
               {\@dblfloat{table}}
               {\end@dblfloat}
%    \end{macrocode}
%
%\qparag{キャプション}
%    \begin{macrocode}
\newlength\abovecaptionskip
\newlength\belowcaptionskip
\setlength\abovecaptionskip{10\p@}
\setlength\belowcaptionskip{0\p@}
\long\def\@makecaption#1#2{%
  \vskip\abovecaptionskip
  \sbox\@tempboxa{#1: #2}%
  \ifdim \wd\@tempboxa >\hsize
    #1: #2\par
  \else
    \global \@minipagefalse
    \hb@xt@\hsize{\hfil\box\@tempboxa\hfil}%
  \fi
  \vskip\belowcaptionskip}
%    \end{macrocode}
%
%^^A----------------
%\subsection{旧式のフォント選択コマンド}
%
%    \begin{macrocode}
\DeclareOldFontCommand{\rm}{\normalfont\rmfamily}{\mathrm}
\DeclareOldFontCommand{\sf}{\normalfont\sffamily}{\mathsf}
\DeclareOldFontCommand{\tt}{\normalfont\ttfamily}{\mathtt}
\DeclareOldFontCommand{\bf}{\normalfont\bfseries}{\mathbf}
\DeclareOldFontCommand{\it}{\normalfont\itshape}{\mathit}
\DeclareOldFontCommand{\sl}{\normalfont\slshape}{\@nomath\sl}
\DeclareOldFontCommand{\sc}{\normalfont\scshape}{\@nomath\sc}
\DeclareRobustCommand*\cal{\@fontswitch\relax\mathcal}
\DeclareRobustCommand*\mit{\@fontswitch\relax\mathnormal}
%    \end{macrocode}
%
%^^A----------------
%\subsection{相互参照}
%
%\qparag{目次}
%    \begin{macrocode}
\newcommand\@pnumwidth{1.55em}
\newcommand\@tocrmarg{2.55em}
\newcommand\@dotsep{4.5}
\setcounter{tocdepth}{2}
\newcommand\tableofcontents{%
    \if@twocolumn
      \@restonecoltrue\onecolumn
    \else
      \@restonecolfalse
    \fi
    \chapter*{\contentsname
        \@mkboth{%
           \MakeUppercase\contentsname}{\MakeUppercase\contentsname}}%
    \@starttoc{toc}%
    \if@restonecol\twocolumn\fi
    }
\newcommand*\l@part[2]{%
  \ifnum \c@tocdepth >-2\relax
    \addpenalty{-\@highpenalty}%
    \addvspace{2.25em \@plus\p@}%
    \setlength\@tempdima{3em}%
    \begingroup
      \parindent \z@ \rightskip \@pnumwidth
      \parfillskip -\@pnumwidth
      {\leavevmode
       \large \bfseries #1\hfil \hb@xt@\@pnumwidth{\hss #2}}\par
       \nobreak
         \global\@nobreaktrue
         \everypar{\global\@nobreakfalse\everypar{}}%
    \endgroup
  \fi}
\newcommand*\l@chapter[2]{%
  \ifnum \c@tocdepth >\m@ne
    \addpenalty{-\@highpenalty}%
    \vskip 1.0em \@plus\p@
    \setlength\@tempdima{1.5em}%
    \begingroup
      \parindent \z@ \rightskip \@pnumwidth
      \parfillskip -\@pnumwidth
      \leavevmode \bfseries
      \advance\leftskip\@tempdima
      \hskip -\leftskip
      #1\nobreak\hfil \nobreak\hb@xt@\@pnumwidth{\hss #2}\par
      \penalty\@highpenalty
    \endgroup
  \fi}
\newcommand*\l@section{\@dottedtocline{1}{1.5em}{2.3em}}
\newcommand*\l@subsection{\@dottedtocline{2}{3.8em}{3.2em}}
\newcommand*\l@subsubsection{\@dottedtocline{3}{7.0em}{4.1em}}
\newcommand*\l@paragraph{\@dottedtocline{4}{10em}{5em}}
\newcommand*\l@subparagraph{\@dottedtocline{5}{12em}{6em}}
%    \end{macrocode}
%
%\qparag{図目次・表目次}
%    \begin{macrocode}
\newcommand\listoffigures{%
    \if@twocolumn
      \@restonecoltrue\onecolumn
    \else
      \@restonecolfalse
    \fi
    \chapter*{\listfigurename
      \@mkboth{\MakeUppercase\listfigurename}%
              {\MakeUppercase\listfigurename}}%
    \@starttoc{lof}%
    \if@restonecol\twocolumn\fi
    }
\newcommand*\l@figure{\@dottedtocline{1}{1.5em}{2.3em}}
\newcommand\listoftables{%
    \if@twocolumn
      \@restonecoltrue\onecolumn
    \else
      \@restonecolfalse
    \fi
    \chapter*{\listtablename
      \@mkboth{%
          \MakeUppercase\listtablename}%
         {\MakeUppercase\listtablename}}%
    \@starttoc{lot}%
    \if@restonecol\twocolumn\fi
    }
\let\l@table\l@figure
%    \end{macrocode}
%
%\qparag{参考文献リスト}
%学位論文では参考文献リストの見出し(つまり ``References'')が
%目次に載るのが正しいらしいので |\addcontentsline| を加えた.
%ちなみに v1.0 でそうならなかったのは,
%\PKN{report}, \PKN{j-report} がそうでないから.
%\changes{v1.1a}{2005/02/24}
%  {thebibliography 中に `addcontentsline 追加.}
%    \begin{macrocode}
\newdimen\bibindent
\setlength\bibindent{1.5em}
\newenvironment{thebibliography}[1]
     {\chapter*{\bibname
        \@mkboth{\MakeUppercase\bibname}{\MakeUppercase\bibname}}%
      \addcontentsline{toc}{chapter}{\bibname}% added(v1.1a)
      \list{\@biblabel{\@arabic\c@enumiv}}%
           {\settowidth\labelwidth{\@biblabel{#1}}%
            \leftmargin\labelwidth
            \advance\leftmargin\labelsep
            \@openbib@code
            \usecounter{enumiv}%
            \let\p@enumiv\@empty
            \renewcommand\theenumiv{\@arabic\c@enumiv}}%
      \sloppy
      \clubpenalty4000
      \@clubpenalty \clubpenalty
      \widowpenalty4000%
      \sfcode`\.\@m}
     {\def\@noitemerr
       {\@latex@warning{Empty `thebibliography' environment}}%
      \endlist}
\newcommand\newblock{\hskip .11em\@plus.33em\@minus.07em}
\let\@openbib@code\@empty
%    \end{macrocode}
%
%\qparag{索引}
%    \begin{macrocode}
\newenvironment{theindex}
               {\if@twocolumn
                  \@restonecolfalse
                \else
                  \@restonecoltrue
                \fi
                \columnseprule \z@
                \columnsep 35\p@
                \twocolumn[\@makeschapterhead{\indexname}]%
                \@mkboth{\MakeUppercase\indexname}%
                        {\MakeUppercase\indexname}%
                \thispagestyle{plain}\parindent\z@
                \parskip\z@ \@plus .3\p@\relax
                \let\item\@idxitem}
               {\if@restonecol\onecolumn\else\clearpage\fi}
\newcommand\@idxitem{\par\hangindent 40\p@}
\newcommand\subitem{\@idxitem \hspace*{20\p@}}
\newcommand\subsubitem{\@idxitem \hspace*{30\p@}}
\newcommand\indexspace{\par \vskip 10\p@ \@plus5\p@ \@minus3\p@\relax}
%    \end{macrocode}
%
%\qparag{脚注}
%なぜここにあるの?
%    \begin{macrocode}
\renewcommand\footnoterule{%
  \kern-3\p@
  \hrule\@width.4\columnwidth
  \kern2.6\p@}
\@addtoreset{footnote}{chapter}
\newcommand\@makefntext[1]{%
    \parindent 1em%
    \noindent
    \hb@xt@1.8em{\hss\@makefnmark}#1}
%    \end{macrocode}
%
%^^A----------------
%\subsection{単語}
%
%\begin{macro}{\contentsname}
%\begin{macro}{\listfigurename}
%\begin{macro}{\listtablename}
%\begin{macro}{\bibname}
%\begin{macro}{\indexname}
%目次・図目次・表目次・参考文献一覧・目次の部に付される見出し.
%    \begin{macrocode}
\newcommand\contentsname{Contents}
\newcommand\listfigurename{List of Figures}
\newcommand\listtablename{List of Tables}
\newcommand\bibname{References}
\newcommand\indexname{Index}
%    \end{macrocode}
%\end{macro}
%\end{macro}
%\end{macro}
%\end{macro}
%\end{macro}
%\begin{macro}{\figurename}
%\begin{macro}{\tablename}
%図(|figure|), 表(|table|)のキャプションで用いられる.
%    \begin{macrocode}
\newcommand\figurename{Figure}
\newcommand\tablename{Table}
%    \end{macrocode}
%\end{macro}
%\end{macro}
%\begin{macro}{\partname}
%\begin{macro}{\chaptername}
%\begin{macro}{\appendixname}
%部(|\part|), 章(|\chapter|)および付録中の章の見出しで用いられる.
%    \begin{macrocode}
\newcommand\partname{Part}
\newcommand\chaptername{Chapter}
\newcommand\appendixname{Appendix}
%    \end{macrocode}
%\end{macro}
%\end{macro}
%\end{macro}
%\begin{macro}{\eabstractname}
%\begin{macro}{\jabstractname}
%英文要旨(|eabstract|)および和文要旨(|jabstract|)の見出し.
%    \begin{macrocode}
\newcommand\eabstractname{Abstract}
\newcommand\jabstractname{\ist@j@abst}
%    \end{macrocode}
%\end{macro}
%\end{macro}
%
%\begin{macro}{\ethesisname}
%\begin{macro}{\jthesisname}
%\begin{macro}{\einterimname}
%\begin{macro}{\jinterimname}
%\begin{macro}{\thesisgrade}
%\begin{macro}{\ist@whatscience}
%表紙の中で用いられる語句.
%v1.1f から |\jinterimname| を空白(|\quad|)から
%「中間報告」に変更した.
%中間報告(要旨提出)の様式には不明な点も多いのだが,
%これが最も正しいことにしてしまおう.
%    \begin{macrocode}
\if@seniorthesis
 \newcommand\ethesisname{A Senior Thesis}
 \newcommand\einterimname{An Interim Report (Abstract)}
 \newcommand\jthesisname{\ist@j@senior}
 \newcommand\jinterimname{\ist@j@interim}
 \newcommand\thesisgrade{Bachelor}
 \newcommand\ist@whatscience{Information Science}
\else \if@masterthesis
 \newcommand\ethesisname{A Master Thesis}
 \newcommand\einterimname{An Interim Report (Abstract)}
 \newcommand\jthesisname{\ist@j@master}
 \newcommand\jinterimname{\ist@j@interim}
 \newcommand\thesisgrade{Master}
 \newcommand\ist@whatscience{Computer Science}
\else \if@doctorthesis
 \newcommand\ethesisname{A Doctor Thesis}
 \newcommand\einterimname{An Interim Report (Abstract)}
 \newcommand\jthesisname{\ist@j@doctor}
 \newcommand\jinterimname{\ist@j@interim}
 \newcommand\thesisgrade{Doctor}
 \newcommand\ist@whatscience{Computer Science}
\fi \fi \fi
%    \end{macrocode}
%\end{macro}
%\end{macro}
%\end{macro}
%\end{macro}
%\end{macro}
%\end{macro}
%
%以下は各自で設定するもの.
%\begin{macro}{\etitle}
%\begin{macro}{\jtitle}
%標題.
%|\@etitle| が英文標題を表し,
%|\etitle{|\meta{str}|}| は |\@etitle| を \meta{str} に定義する.
%他のコマンドも同様.
%    \begin{macrocode}
\def\etitle#1{\gdef\@etitle{#1}}
\def\jtitle#1{\gdef\@jtitle{#1}}
%    \end{macrocode}
%\end{macro}
%\end{macro}
%\begin{macro}{\eauthor}
%\begin{macro}{\jauthor}
%著者名.
%    \begin{macrocode}
\def\eauthor#1{\gdef\@eauthor{#1}}
\def\jauthor#1{\gdef\@jauthor{#1}}
%    \end{macrocode}
%\end{macro}
%\end{macro}
%\begin{macro}{\esupervisor}
%\begin{macro}{\jsupervisor}
%指導教官名.
%    \begin{macrocode}
\def\esupervisor#1{\gdef\@esupervisor{#1}}
\def\jsupervisor#1{\gdef\@jsupervisor{#1}}
%    \end{macrocode}
%\end{macro}
%\end{macro}
%\begin{macro}{\supervisortitle}
%指導教官の職名.
%    \begin{macrocode}
\def\supervisortitle#1{\gdef\@supervisortitle{#1}}
%    \end{macrocode}
%\end{macro}
%\begin{macro}{\@etitle}
%\begin{macro}{\@jtitle}
%これらの項目が未設定だとエラーにする.
%    \begin{macrocode}
\def\@etitle{\ist@err@notdefd\etitle}
\def\@jtitle{\ist@err@notdefd\jtitle}
\def\@eauthor{\ist@err@notdefd\eauthor}
\def\@jauthor{\ist@err@notdefd\jauthor}
\def\@esupervisor{\ist@err@notdefd\esupervisor}
\def\@jsupervisor{\ist@err@notdefd\jsupervisor}
\def\@supervisortitle{\ist@err@notdefd\supervisortitle}
%    \end{macrocode}
%\end{macro}
%\end{macro}
%
%\begin{macro}{\@date}
%\begin{macro}{\today}
%日付が指定されてないとエラーにする.
%ただし |\today| は有効である.
%\changes{v1.1a}{2005/02/24}
%  {\@date の初期値をエラー発生に変更.}
%    \begin{macrocode}
\def\@date{\ist@err@notdefd\@date}
\def\today{\ifcase\month\or
  January\or February\or March\or April\or May\or June\or
  July\or August\or September\or October\or November\or December\fi
  \space\number\day, \number\year}
%    \end{macrocode}
%\end{macro}
%\end{macro}
%
%\begin{macro}{\supervisortitleline}
%\begin{macro}{\@supervisortitleline}
%\begin{macro}{\thesupervisortitle}
%指導教官の職名の行の全体.
%この指定の中で |\@supervisortitle| を参照する必要が
%あるので, これを |\thesupervisortitle| として表に出しておく.
%この項目の初期値は ``|\@supervisortitle| of |\ist@whatscience|''
%で, |\ist@whatscience| は ``Information Science'' (|senior|)
%または ``Computer Science'' (|master|/|doctor|) としている.
%本当はどうするのが正しいのだろう?
%    \begin{macrocode}
\newcommand\thesupervisortitle{\@supervisortitle}
\newcommand*\supervisortitleline[1]{\gdef\@supervisortitleline{#1}}
\newcommand\@supervisortitleline{%
  \@supervisortitle\ of \ist@whatscience
}
%    \end{macrocode}
%\end{macro}
%\end{macro}
%\end{macro}
%
%\begin{macro}{\ist@j@senior}
%\begin{macro}{\ist@j@master}
%\begin{macro}{\ist@j@doctor}
%\begin{macro}{\ist@j@abst}
%\begin{macro}{\ist@j@interim}
%\qparag{和文語句}
%欧文用 \TeX\ で通すという無理をするために,
%ちょっと |\catcode| している.
%気にしてはいけない.
%    \begin{macrocode}
\ifist@english \catcode`\.=14 \else \catcode`\.=9 \fi
.\newcommand\ist@j@senior{卒業論文}
.\newcommand\ist@j@master{修士論文}
.\newcommand\ist@j@doctor{博士論文}
.\newcommand\ist@j@abst{論文要旨}
.\newcommand\ist@j@interim{中間報告}
.\newcommand\ist@jparen[1]{(#1)}
\catcode`\.=12\relax
%    \end{macrocode}
%\end{macro}
%\end{macro}
%\end{macro}
%\end{macro}
%\end{macro}
%
%^^A----------------
%\subsection{初期化}
%
%\LaTeX\ のいくつかの命令を無効にする.
%    \begin{macrocode}
\def\title{\ist@err@invalid\title}
\def\author{\ist@err@invalid\author}
\def\and{\ist@err@invalid\and}
\def\abstract{\ist@err@invalid\abstract}
%    \end{macrocode}
%
%欧文 \TeX\ 使用時は |ist-en.clo| を読み込む.
%    \begin{macrocode}
\if e\ist@engine
  \input{ist-en.clo}
\fi
%    \end{macrocode}
%
%|\sloppy| の定義をかなり sloppy になるように直した.
%|sloppy| オプションが指定されているならば |\sloppy| にする.
%残りは \PKN{report} のまま.
%    \begin{macrocode}
\setlength\columnsep{10\p@}
\setlength\columnseprule{0\p@}
\pagestyle{plain}
\pagenumbering{arabic}
\def\sloppy{\tolerance 9999 \hbadness 5000
            \emergencystretch 3em
            \hfuzz 2.5\p@ \vfuzz .5\p@}
\ifist@sloppy
  \sloppy
\fi
\if@twoside
\else
  \raggedbottom
\fi
\if@twocolumn
  \twocolumn
  \sloppy
  \flushbottom
\else
  \onecolumn
\fi
%    \end{macrocode}
%
%^^A----------------
%\subsection{終了}
%お疲れ様でした. (誰にいってるの?)
%
%    \begin{macrocode}
%</!isten>
%    \end{macrocode}
%
%
%^^A========================================================
%\section{クラスオプションファイル ist-en.clo}
%\label{sec:Ist-en}
%^^A--------------------------------------------------------
%
%\noindent
%警告: この節の内容は, 読者の精神に影響を与えるような
%表現を含みます.
%\par\medskip
%
%このソースをオプション `|isten|' 付きで \DCN{docstrip} で
%処理すると, ファイル |ist-en.clo| が得られる.
%これを用意しておくと, 欧文用の \LaTeX\ で
%(表紙部と要旨に和文文字が入ったままの)論文のソースが
%コンパイル可能となる
%(|english| オプション指定時と同じ出力).
%ただし, この機能は実験的なものであり,
%必ずしも正しく動作する保証はない.
%
%\noindent ※ 制限事項:\quad
%|\jauthor| 等のコマンドの場合, 以降に出現する
%最初の ``|}| (+ 空白文字) + 改行'' の中の |}| を引数の
%終わりを示す |}| と見なす.
%これが実際と相違する場合には正しく動作しない.
%特に SJIS の場合, 和文文字 2 バイト目の 7D$_{16}$ が |}| と
%認識されるので注意.
%|\begin{jabstract}| に関しては, 以降の最初の |\end{jabstract}| の
%出現を終端とし, |verbatim| と同じ制限がかかる.
%
%    \begin{macrocode}
%<*isten>
\ProvidesFile{ist-en}
    [2005/12/25 v1.1f
     Class option file]
%
\def\ist@makesjenv#1{%
  \@namedef{#1}{\ist@sj@gengobbler{#1}\ist@sj@begin}%
  \expandafter\let\csname end#1\endcsname=\ist@sj@end}
\def\ist@makesjcmd#1{\let#1=\ist@sj@cbegin}
\def\istallowesccode{\catcode`\^^[=9 }
\def\istdisallowesccode{\catcode`\^^[=15 }
%
\begingroup \catcode`\|=0 \catcode`\[=1 \catcode`\]=2 %
 \catcode`\^^M=12 \catcode`\{=12 \catcode`\}=12 \catcode`\\=12 %
 |gdef|ist@sj@gengobbler#1[%
   |def|ist@sj@gobble##1\end{#1}[|end[#1]]]%
 |gdef|ist@sj@cgobble#1}^^M[|ist@sj@cend]%
|endgroup
\def\ist@sj@begin{\ist@sj@sanitize \ist@sj@gobble}
\def\ist@sj@end{}
\def\ist@sj@cbegin{\begingroup \ist@sj@sanitize \ist@sj@cgobble}
\def\ist@sj@cend{\endgroup}
\def\ist@sj@sanitize{\let\do\@makeother\dospecials
  \catcode`\^^M=12 \catcode`\ =9 \catcode`\^^[=9 }
%
\ist@makesjenv{jabstract}
\ist@makesjcmd{\jsupervisor}
\ist@makesjcmd{\jtitle}
\ist@makesjcmd{\jauthor}
\istallowesccode
%</isten>
%    \end{macrocode}
%
%^^A--------------------------------------------------------
%\Finale
%
%^^A# End of file
