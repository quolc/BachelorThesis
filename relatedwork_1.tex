\subsection{Embedding Information in Audio Signals}
% 電子透かしについての基本的な歴史
Techniques for embedding additional data in digital contents such as image and audio in noise-tolerable and imperceptible form are called {\it digital watermarking}, and they have their origin in steganography, or data-hiding technologies.
The word {\it digital watermark} or {\it electronic watermark} was used first in 1992 in \cite{tirkel1993electronic} for a technique to embed data in a bitmap image.

% 音響透かしの初期の研究事例
The fundamentals of audio watermarking were developed in the late 1990's.
In 1996, Bender, Gruhl, Morimoto and Lu presented basic techniques for audio watermarking such as low-bit coding, phase coding, spread spectrum and echo hiding \cite{bender1996techniques}.
After their pioneer work, many researchers have been working for enhancing audio watermarking technique.
Kiah et al. reviewed existing watermarking schemes and summarized their advantages and disadvantages in 2011 \cite{mat2011review}.
% 現在は幾つかの手法が代表的に使われている
An elaborated spread spectrum technique with adequate robustness and detection difficulty that employs a large range of frequency has been researched by Cox et al. \cite{cox1997secure,cox2001digital}.
Echo-hiding is also a widely studied watermarking technique that represents binary data by introducing imperceptible echoes at certain timings in a audio signal. Oh et al. \cite{oh2001new} and Ko, Nishimura and Suzuki \cite{ko2005time} presented improved echo-hiding techniques that have high durability for attempts to remove.
% 認知特性を利用している
In order to improve their transparency, most of audio watermarking techniques exploit perceptual characteristics of human auditory system (HAS) like auditory masking, phenomenon that audibility of a sound is weakened by the presence of another sound.
% frequency domainではFMが、time domainではTMが主に利用されている

% 現在の主な音声透かしの使用例
Since the appearance of the technique, the main usage of audio watermarks has been
%limited to some applications such as
protecting contents from abuses by identifying copyright information and source tracking of audio contents.
%, which only require a small amount of information to embed.

\subsection{Acoustic Communication between Devices}
%放送 ... Acoustic OFDM
Watermarking techniques have also been used to transmit information to consumer devices in entertainment and digital signage for enhancing user experience.
Superimposing text information such as URLs and artist information of songs on audio broadcasting is a typical example of such usage.
Users hearing a radio show can extract embedded watermarks from the broadcasted sound using their devices like smartphones and access online contents related to the program.
Researchers at NTT DoCoMo developed a watermarking framework with high bit rate named {\it acoustic OFDM} \cite{matsuoka2008acoustic}. YAMAHA also developed a similar technology called {\it INFOSOUND} \cite{infosound}, and it was used in commercial radio broadcasting.

%体験 ... Museum Navigation, Cryptone
Museum installations and live concerts have possible demand for this kind of device communication, because it can improve interactivity of exhibitions and performance, and only requires common audio devices like loud speakers.
Gebbensleben, Dittmann and Vielhauer. presented a museum guide system using audio watermarking technology \cite{gebbensleben2006multimodal}, which uses visitor's personal devices to receive information about objects and show them.
Hirabayashi and Shimizu developed a system called {\it Cryptone} that enables interaction between a live performer and audience at venues for musical performances using high frequency acoustic dual-tone multi-frequency (DTMF) signals like our work \cite{Hirabayashi:2012:CIP:2407707.2407712}.

% 位置検出
Nakashima, Tachibana and Babaguchi proposed a unique method to estimate the position of a camcorder at a theater by watermarking the soundtrack of a movie \cite{nakashima2009watermarked}.
The position of a camera that recorded a movie surreptitiously in a theater can be estimated from the watermarks in the recorded video, since distances between the camera and each speakers of the theater make a distinctive pattern of watermarks in the time series of the video.

\subsection{Real-time Audio Watermarking}
% どれかReal-time watermarkingの技術一つ
Though most of the watermarking schemes are designed for applying to static contents, some techniques for embedding watermarks simultaneously with the host signal being performed are studied recently. They are called {\it real-time watermarking}.
% Live performance
Tachibana presented a real-time watermarking scheme named {\it sonic watermarking} for watermarking musical live performance \cite{tachibana2003audio}. In his scheme, host signal and watermark signal generated by analyzing the host signal are played separately from two speakers and mixed in the air in real-time.
Yamamoto and Iwakiri also proposed a real-time watermarking technique specialized in musical performance with less delay between host signal and watermark signal by modifying the wavetables of electronic musical instruments \cite{yamamoto2010real}.
Though these techniques realized real-time embedding of watermarking in live performance, they can not used in a situation without an amplifiers because sound signals must be analyzied or modified before being played to audience.
