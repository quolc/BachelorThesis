% 提案手法の有用性を検証するために、リアルなハードウェア構成でwatermarking schemeの性能評価実験を行った
To show the practicality of proposed system, we conducted a series of performance evaluations on our watermarking scheme using a sample hardware setup assumed real usage.
% ハードウェア構成
% 家庭用ビデオカメラ(型番)とスマートフォン(型番)で構成され、スマートフォンはカメラの外部マイクに固定する
This consisted of a consumer-use digital camcorder (SONY NEX-VG30H) and a smartphone (SAMSUNG Galaxy S) to generate and transmit watermarks.
The phone was attached to the microphone unit of the camera.
% カメラ設定
In the evaluations, movies were recorded in the conventional MPEG-2 format using standard definition image quality, and sound was recorded in 2ch stereo PCM format.
Default values were used for the other configurations of the camcorder.
% スマートフォン設定
Annotation transmitter applications were installed on the smartphone.
The audio output level of the phone was set an appropriate value empirically decided.

\begin{figure}[htbp]
 \begin{center}
  \vspace{5mm}
  \includegraphics[width=100mm]{evaluation_environment.pdf}
 \end{center}
 \caption{The hardware setup used in the performance evaluations.}
 \label{fig:eval_hard}
\end{figure}
