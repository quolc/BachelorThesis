\if0
4-4. Acoustic DTMF in Inaudible Frequency Range
	pg2. データフレームの定義
	pg3. パケットの構造(データのエンディアンも)
	※ CRCビットによる誤り検出手法を定める
\fi

% DTMFに関して多少の詳細説明
An watermark of AnnoTone is modulated using DTMF technique mentioned above.
It can contain any kinds of annotation data such as integer value, floating-point number, character string and set of them.
Annotation data is serialized to a byte array and converted to a watermark packet structured as below.

% データフレームの定義
A data frame is the minimum unit of data representation in watermarking scheme of AnnoTone, which is a DTMF signal composed from two sinusoidal waves of the seven sub-carrier frequencies with length of 10 ms.
Since each data frame represents 4-bits information from 0 to 15, a data of one byte is encoded into two data frames.
The former frame represents lower four bits and the latter represents higher four bits of the data.

% ToDo: パケット構造の模式図を描く

% パケットの定義
A packet is a set of several successive data frames representing an annotation data, and can be regarded identical to a watermark.
The data frame at the head of a packet is called start frame. It is fixed to represent ``2'', and it functions only as a sign of the start position of the packet. There is no special reason for choosing this number.
The second data frame specifies the size of payload following it to be $ 2^n $ bytes where $n$ is the value of this data frame. It enables a payload size to vary depending on the type of an annotation. Consequently, the size of a packet can also vary.
From the third data frame, a payload region containing the body of an annotation data starts and continues for the length specified in the second data frame.
Multi-byte data such as integer values are ordered in little endian, and multiple data are arranged in their original order in a payload region.
For example, an array of two short integer values \{0x1234, 0x5678\} would be arranged in a payload region as \{4, 3, 2, 1, 8, 7, 6, 5\}.
The data frame after the end of a payload region represents the annotation type id from 0 to 15 to distinguish multiple kinds of annotations embedded in one media file.
An annotation type id for a watermark can be decided arbitrarily by annotating applications.
A packet ends with two data frames representing the CRC-8 checksum of the packet for error detection in decoding. The polynomial for calculating CRC-8 checksum is $x^8 + x^7 + x^6 + x^4 + x^2 + 1$.
